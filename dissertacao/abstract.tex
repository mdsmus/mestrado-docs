\chapter*{Abstract}
\label{cha:abstract}

Contours can be understood as the shape or format of a object. In
Music contour can represent a parameter in function of another, like
pitch in function of density or density in function of
amplitude. Contours are important because, as well as notes sets and
motives, they can help giving coherence to a musical piece.

Theories of contours have been used in areas such as Musical
Perception and Analysis, but the systematic use of contours for
generation of material composicional matter is still lacking
literature.

In this thesis I present the piece \obra{} [\opus{Around the
  pomegranate}] and its analysis. This piece, for woodwind quintet,
was composed using contour operations combinations associated with
parameters such as pitch, tempo, density and texture. For this job I
did a literature review of contour theories, I did a mapping of
contours to musical elements, I composed studies of possibilities for
experimentation with contours, I develop the Goiaba, a software to
assist in processing contours for composition, and finally composed
the piece \obra{}.

This study helps to elevate the state of art of contour theories by
composition contour operations experiments, and helps with new tools
to the area of composition.
%% Conclude that the
I conclude that contours can be used in a systematic way in musical
composition, but we still need further study. Thus this depth and
continuity in development of Goiaba are possible future activities
arising this work.
