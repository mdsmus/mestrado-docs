\chapter*{Abstract}
\label{cha:abstract}

Contour can be understood as the shape ou format of a object. In Music
contour can map a parameter as a function of another, like pitch as
function of density or density as function of amplitude. Contours are
important because, as well as notes sets and motivesm they can help
giving coherence to a musical piece.

Theories of contours have been used in areas such as Musical
Perception and Analysis, but the systematic use of contours for
generation of material composicional matter is still lacking
literature.

In this paper I present the piece for woodwind quintet \obra{},
composed upon contour operations combinations associated with
parameters such as pitch, tempo, density and texture. For this job I
did a literature review of contour theories, I did a mapping of
contours to musical elements, I composed studies of possibilities for
experimentation with contours, I develop the Goiaba, a software to
assist in processing contours for composition, and finally composed
the piece \obra{}.

This study helps to elevate the state of art of contour theories and
helps with new tools to the area of composition.
%% Conclude that the
I conclude that contours can be used in a systematic way in musical
composition, but we still need further study. Thus this depth and
continuity in development of Goiaba are possible future activities
arising this work.
