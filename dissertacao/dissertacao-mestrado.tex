\documentclass[12pt,brazil]{book}
\usepackage[utf8x]{inputenc}
\usepackage{babel}
\usepackage[a4paper,top=2.5cm,left=2.5cm,right=2.5cm,bottom=2.5cm]{geometry}
\usepackage{graphicx}
\usepackage{url}
\usepackage{achicago}
\usepackage{setspace}
\usepackage{amsmath}

\doublespacing

% usar para citação integral indentada com tradução
\newcommand{\citacaoindt}[4]{
  \begin{quote}
    \normalsize
%    \selectlanguage{english}
    {#2}\footnote{
      \selectlanguage{brazil}
      ``{#1}''.
    }.
    \selectlanguage{brazil}
    \cite[#3]{#4}.
  \end{quote}
}

\title{Contornos melódicos para composição}
\author{Marcos da Silva Sampaio}
\begin{document}
\maketitle
\tableofcontents

\chapter{Introdução}
\label{cha:introducao}

A idéia de preservação de contorno é encontrada em diferentes
situações musicais, como nas respostas reais e tonais das fugas
\cite[p. 29]{morris87:composition}.

\chapter{Sobre contornos}
\label{cha:sobre-contornos}

%% brainstorm

% análise: comparação e equivalência de materiais.
% análise: objetivos diferentes.
% redução: tipologia de adams e algoritmo de morris
% equivalência a partir de matriz de comparação
% cognição: dowling
% terminologias diferentes: friedmann
% problemas com teorias: clifford
% estruturação da composição a partir de contornos
% exemplos com análise de peças?
% contornos: coerência, schoenberg

A terminologia utilizada em teorias de contornos melódicos não é
uniforme. Cada autor utiliza termos diferentes para ferramentas
semelhantes \cite{friedmann87:response}. A tabela
\ref{tab:compara-ferramentas} tem uma comparação dos termos usados por
Friedmann, e Morris, Marvin e Laprade.

\begin{table}
  \centering
  \begin{tabular}{l|l}
    Friedmann & Marvin e Laprade \\ \hline
    Contour Adacency Series (CAS) & INT \\
    Contour Class (CC) & cseg \\
    Subset & csubseg 
  \end{tabular}
  \caption{Quadro comparativo de ferramentas de análise de contornos}
  \label{tab:compara-ferramentas}
\end{table}

A Matriz de Comparação representa o total de comparações de um
contorno. É obtida através da comparação de todos os pares ordenados
de um contorno. Em uma matriz $E$ de um contorno $P$, a posição $E
(x,y)$ contém $Com (Px,Py)$. A matriz exibe uma simetria de sinais
invertidos em torno da diagonal principal, preenchida apenas por
zeros. Dois contornos são equivalentes se geram a mesma matriz de
comparação \cite[p. 28]{morris87:composition}. Um exemplo desta matriz
referente ao contorno $\langle 4\:3\:5\:6 \rangle$ pode ser visto na
tabela \ref{tab:matriz-4356}.

\begin{table}
  \centering
  \begin{tabular}{r|cccc}
    & $4$ & $3$ & $5$ & $6$ \\
    \hline
    $4$ & $0$ & $-$ & $+$ & $+$ \\
    $3$ & $+$ & $0$ & $+$ & $+$ \\
    $5$ & $-$ & $-$ & $0$ & $+$ \\
    $6$ & $-$ & $-$ & $-$ & $0$ \\
  \end{tabular}
  \caption{Matriz de comparação do contorno $\langle 4\:3\:5\:6 \rangle$}
\label{tab:matriz-4356}
\end{table}

\section{Percepção de contornos}
\label{sec:perc-de-cont}

Contorno melódico é uma importante característica musical no
reconhecimento de melodias familiares
\cite[p. 136]{dowling.ea86:music}. Esta afirmação é endossada por
experimentos realizados por White \cite{white60:recognition}, e
Dowling e Fujitani \cite{dowling.ea71:contour}. Estes experimentos
consistiam em submeter ouvintes ao reconhecimento de versões de
canções familiares às quais intervalos melódicos e/ou ritmo eram
modificados e o contorno preservado.

De acordo com Friedmann as implicações do estudo de contornos na
música do século XX são mais significativas para os ouvintes do que
para os compositores. Isto porque a percepção de contornos é mais
geral do que a percepção de altura e dos outros elementos do universo
da teoria atonal \cite[p. 224]{friedmann85:methodology}.

A idéia de Hindemith sobre percepção de contornos é semelhante à dos
demais autores, embora o contexto em que se aplique seja o do estudo
da harmonia. De acordo com ele é mais fácil lembrar de sucessão
rítmica e de curvas de uma linha melódica do que de diferenças em
tensão entre harmonias \cite[p. 175]{hindemith41:craft}.

\section{Contorno como determinante composicional}
\label{sec:cont-como-determ}

Clifford afirma que

\citacaoindt{contour, in absence of other pervasive systems of pitch
  organization, represents a structural factor equal in significance
  to pitch or set class relations}{contorno, na falta de outros
  sistemas ocupantes de organização de altura, representa um fator
  estrutural igual em significado a relações de alturas ou de classes
  de conjuntos}{p. 157}{clifford95:contour}

No nível melódico relações de contorno podem associar segmentos de
contorno distintos entre dois ou mais segmentos. Neste nível contorno
pode contribuir com um processo de transformação melódica
\cite[p. 159]{clifford95:contour}


\chapter{Ferramentas}
\label{cha:ferramentas}

\chapter{Análise da composição}
\label{cha:anal-da-comp}

\chapter{Conclusão}
\label{cha:conclusao}

%%% bibliografia
%% teoria dos contornos

\bibliographystyle{mdschicago}
\bibliography{melodic-contour,music-perception,composition}

\chapter{Apêndice}
\label{cha:apendice}

\section{Glossário}
\label{sec:glossario}

\paragraph{Contour Adjacency Series (CAS)}
\label{sec:cont-adjac-seri}

Série de elementos adjacentes de um contorno. É uma série de sinais
'+' e '-' referentes às inclinações positivas e negativas entre notas
adjascentes de um contorno. Contornos adjacentes de mesma altura são
desconsiderados, assim como não há uma indicação de repetição de
altura em larga escala. Por exemplo, para um contorno \texttt{d c f
  d}, CAS = \texttt{$<$-,+,-$>$}; para um contorno \texttt{d c c e}, CAS =
\texttt{$<$-,+$>$}.

A principal utilidade da CAS está na comparação de equivalência com
retrógrado ou rotação de ordem dos elementos do contorno.

\paragraph{Contour Class (CC)}
\label{sec:contour-class-cc}

Classe de contornos. É uma série ordenada de números que representa a
relação de alturas entre todas as notas de um contorno, na qual a nota
mais grave é indicada por 0 e a nota mais aguda por n-1, onde n=total
de notas. Por exemplo, para um contorno \texttt{d c f e}, CC = \texttt{$<$1 0
3 2$>$}.

\paragraph{Contour Adjacency Series Vector (CASV)}
\label{sec:cont-adjac-seri-1}

Vetor de classe de contornos adjacentes. É um vetor de dois dígitos
com a soma das inclinações positivas e negativas de um contorno. Por
exemplo, para um contorno \texttt{d c f e} CASV = \texttt{$<$1,2$>$}.

Sua principal utilidade é estabelecer uma classe de equivalência entre
rotações, retrógrados e outras permutações de ordem de elementos.

\paragraph{Contour Interval (CI)}
\label{sec:contour-interval-ci}

Intervalo de contorno. É a distância entre dois elementos de uma
classe de contorno (CC). Por exemplo, em CC $<$1 0 3 2$>$, a CI de \texttt{1-0} é
\texttt{-1}, a CI de \texttt{0-3} é \texttt{+3}.

\paragraph{Contour Interval Succession (CIS)}
\label{sec:cont-interv-succ}

Sucessão de intervalos de contorno é uma série de números que
representam os intervalos de contorno (CI) de uma classe de contorno
(CC). Por exemplo, em CC \texttt{$<$1 0 3 2$>$}, CIS = \texttt{$<$1 3 -1$>$}.

\paragraph{Contour Interval Array (CIA)}
\label{sec:cont-interv-array}

Vetor de intervalos de contorno. É uma série de números que representa
a multiplicidade de intervalos de contornos (CI) de uma classe de
contorno (CC). Esta série é dividida em duas e cada número representa
o número de CI's de valor \texttt{1}, \texttt{2}, \texttt{3}, etc, a
depender do tamanho do contorno. A primeira parte da série representa
CI's positivos e a segunda representa CI's negativos. Por exemplo,
para CC = \texttt{$<$1 0 3 2$>$}, CIA = \texttt{$<$1 2 1$>$,$<$2 0
  0$>$}. Esta CC tem um intervalo de valor \texttt{+1}, dois de valor
\texttt{+2}, um de valor \texttt{+3}, dois de valor \texttt{-1} e
nenhum de valores \texttt{-2} e \texttt{-3}.

\paragraph{Contour Class Vector I (CCVI)}
\label{sec:contour-class-vector-1}

Vetor de classe de contorno I. É um par de números que representa o
grau de ascendência e descendência de uma classe de contorno (CC). É
obtido através da soma da multiplicação do número de ocorrências de
cada intervalo de contorno (CI) pelo próprio valor do intervalo. Por
exemplo, em um CIA \texttt{$<$1 2 1$>$,$<$2 0 0$>$} CCVI =
\texttt{$<$8 2$>$}.

\paragraph{Contour Class Vector II (CCVII)}
\label{sec:contour-class-vector-2}

Vetor de classe de contorno II. É um par de números que representa o
grau de ascendência e descendência de uma classe de contorno (CC). É
obtido através da soma dos valores positivos e negativos das
ocorrências de intervalos de contorno (CI). Por exemplo, em um CIA
\texttt{$<$1 2 1$>$,$<$2 0 0$>$}, CCVII = \texttt{$<$8 2$>$}

\end{document}
