\documentclass[12pt,brazil]{book}
\usepackage[utf8x]{inputenc}
\usepackage{babel}
\usepackage[a4paper,top=2.5cm,left=2.5cm,right=2.5cm,bottom=2.5cm]{geometry}
\usepackage{graphicx}
\usepackage{url}
\usepackage{achicago}

\title{Contornos melódicos para composição}
\author{Marcos da Silva Sampaio}
\begin{document}
\maketitle

\chapter{Introdução}
\label{cha:introducao}

\chapter{Sobre contornos}
\label{cha:sobre-contornos}

A terminologia utilizada em teorias de contornos melódicos não é
uniforme. Cada autor utiliza termos diferentes para ferramentas
semelhantes \cite{friedmann87:response}. A tabela
\ref{tab:compara-ferramentas} tem uma comparação dos termos usados por
Friedmann \cite{friedmann85:methodology}, e Marvin e Laprade
\cite{marvin.ea87:relating}.

\begin{table}
  \centering
  \begin{tabular}{l|l}
    Friedmann & Marvin e Laprade \\ \hline
    Contour Adacency Series (CAS) & INT \\
    Contour Class (CC) & cseg
  \end{tabular}
  \caption{Quadro comparativo de ferramentas de análise de contornos}
  \label{tab:compara-ferramentas}
\end{table}
 
\chapter{Ferramentas}
\label{cha:ferramentas}

\chapter{Análise da composição}
\label{cha:anal-da-comp}

\chapter{Conclusão}
\label{cha:conclusao}

%%% bibliografia
%% teoria dos contornos

\bibliographystyle{achicago}
\bibliography{melodic-contour}

\end{document}
