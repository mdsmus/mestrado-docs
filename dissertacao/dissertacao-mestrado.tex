\documentclass[12pt,brazil]{book}
\usepackage[utf8x]{inputenc}
\usepackage{babel}
\usepackage[a4paper,top=2.5cm,left=2.5cm,right=2.5cm,bottom=2.5cm]{geometry}
\usepackage{graphicx}
\usepackage{url}
\usepackage{setspace}
\usepackage{amsmath}
\usepackage{subfigure}
\usepackage{glossaries}

\makeglossaries
\glossarystyle{altlistgroup}

\newglossaryentry{cas}{
  name=Contour Adjacency Series,
  description={
    Série de elementos adjacentes de um contorno. É uma série de sinais
    '+' e '-' referentes às inclinações positivas e negativas entre notas
    adjascentes de um contorno. Contornos adjacentes de mesma altura são
    desconsiderados, assim como não há uma indicação de repetição de
    altura em larga escala. Por exemplo, para um contorno \texttt{d c f
      d}, CAS = \texttt{$<$-,+,-$>$}; para um contorno \texttt{d c c e}, CAS =
    \texttt{$<$-,+$>$}.
    A principal utilidade da CAS está na comparação de equivalência com
    retrógrado ou rotação de ordem dos elementos do contorno.
  }
} 

\newglossaryentry{cc}{
  name=Contour Class,
  description={
    Classe de contornos. É uma série ordenada de números que representa a
    relação de alturas entre todas as notas de um contorno, na qual a nota
    mais grave é indicada por 0 e a nota mais aguda por n-1, onde n=total
    de notas. Por exemplo, para um contorno \texttt{d c f e}, CC = \texttt{$<$1 0
      3 2$>$}.
  }
}

\newglossaryentry{casv}{
  name=Contour Adjacency Series Vector,
  description={
    Vetor de classe de contornos adjacentes. É um vetor de dois dígitos
    com a soma das inclinações positivas e negativas de um contorno. Por
    exemplo, para um contorno \texttt{d c f e} CASV = \texttt{$<$1,2$>$}.
    Sua principal utilidade é estabelecer uma classe de equivalência entre
    rotações, retrógrados e outras permutações de ordem de elementos.
  }
}

\newglossaryentry{ci}{
  name=Contour Interval,
  description={
    Intervalo de contorno. É a distância entre dois elementos de uma
    classe de contorno (CC). Por exemplo, em CC $<$1 0 3 2$>$, a CI de \texttt{1-0} é
    \texttt{-1}, a CI de \texttt{0-3} é \texttt{+3}.
  }
}

\newglossaryentry{cis}{
  name=Contour Interval Succession,
  description={
    Sucessão de intervalos de contorno é uma série de números que
    representam os intervalos de contorno (CI) de uma classe de contorno
    (CC). Por exemplo, em CC \texttt{$<$1 0 3 2$>$}, CIS = \texttt{$<$1 3 -1$>$}.
  }
}

\newglossaryentry{cia}{
  name=Contour Interval Array,
  description={
    Vetor de intervalos de contorno. É uma série de números que representa
    a multiplicidade de intervalos de contornos (CI) de uma classe de
    contorno (CC). Esta série é dividida em duas e cada número representa
    o número de CI's de valor \texttt{1}, \texttt{2}, \texttt{3}, etc, a
    depender do tamanho do contorno. A primeira parte da série representa
    CI's positivos e a segunda representa CI's negativos. Por exemplo,
    para CC = \texttt{$<$1 0 3 2$>$}, CIA = \texttt{$<$1 2 1$>$,$<$2 0
      0$>$}. Esta CC tem um intervalo de valor \texttt{+1}, dois de valor
    \texttt{+2}, um de valor \texttt{+3}, dois de valor \texttt{-1} e
    nenhum de valores \texttt{-2} e \texttt{-3}.
  }
}

\newglossaryentry{ccvi}{
  name=Contour Class Vector I,
  description={
    Vetor de classe de contorno I. É um par de números que representa o
    grau de ascendência e descendência de uma classe de contorno (CC). É
    obtido através da soma da multiplicação do número de ocorrências de
    cada intervalo de contorno (CI) pelo próprio valor do intervalo. Por
    exemplo, em um CIA \texttt{$<$1 2 1$>$,$<$2 0 0$>$} CCVI =
    \texttt{$<$8 2$>$}.
  }
}

\newglossaryentry{ccvii}{
  name=Contour Class Vector II,
  description={
    Vetor de classe de contorno II. É um par de números que representa o
    grau de ascendência e descendência de uma classe de contorno (CC). É
    obtido através da soma dos valores positivos e negativos das
    ocorrências de intervalos de contorno (CI). Por exemplo, em um CIA
    \texttt{$<$1 2 1$>$,$<$2 0 0$>$}, CCVII = \texttt{$<$8 2$>$}
  }
}
%%% glossário de marvin.ea87:relating

\newglossaryentry{com-matrix}{
  name=COM-Matrix,
  description={
    Matriz de comparação. Uma matriz bidimensional que mostra os
    resultados da função de comparação $COM(a,b)$ para qualquer altura
    de contorno \eng{c-pitch} no espaço de contorno \eng{c-space}. Se
    $b > a$ a função retorna ``$+$''. Se $b = a$ a função retorna
    $0$. Se $b < a$ a função retorna ``$-$''.
  }
}

\newglossaryentry{cps}{
  name=c-pitch,
  description={
    Altura de contorno. Elementos no espaço de contorno numerados de
    forma ordenada do mais grave para o mais agudo, de $0$ a $(n -
    1)$, em que n representa o número de elementos.
  } 
}

\newglossaryentry{c-space}{
  name=Contour Space,
  description={
    Espaço de contorno. Um tipo de espaço musical consistindo de
    elementos organizados do grave para o agudo desconsiderando os
    intervalos exatos entre os elementos.
  }
}

\newglossaryentry{cseg}{
  name=c-segment,
  description={
    Segmento de contorno. Um conjunto ordenado de alturas de contorno
    no espaço de contorno \eng{\gls{c-space}}.
  }
}

\newglossaryentry{csegclass}{
  name=c-space segment class,
  description={
    Classe de segmentos do espaço de contorno. Uma classe de
    equivalência feita de todos os segmentos de contorno relacionados
    por identidade, translação, retrogradação, inversão e
    retrogradação da inversão.
  }
}

\newglossaryentry{csubseg}{
  name=c-subsegment,
  description={
    Subsegmento de contorno. Qualquer subgrupo de um dado segmento de
    contorno. Pode ser compreendido de alturas de contornos adjacentes
    ou não do segmento de contorno original.
  }
}

\newglossaryentry{p-space}{
  name=Pitch Space,
  description={
    Espaço de altura.
  }
}

\newglossaryentry{int}{
  name=INT,
  symbol={$INT_n$},
  description={
    Qualquer das diagonais à direita da diagonal principal da
    \gls{com-matrix}.
  }
}

\newglossaryentry{I}{
  name=Inversão,
  description={
    Inversão de um \gls{cseg} S constituído de $n$ distintos \gls{cps}.
    É calculada subtraindo cada \gls{cps} de $(n-1)$.
  }
}

\newglossaryentry{normalform}{
  name={Forma Normal},
  description={
    Um vetor ordenado no qual elementos de uma \gls{cseg} de $n$
    distintos \gls{cps} são numerados de 0 a $(n-1)$ e listados na
    ordem temporal
  }
}

\newglossaryentry{T}{
  name=Translação,
  description={
    Operação através da qual um \gls{csubseg} é renumerado de $0$
    (nota mais grave) a $(n-1)$ (nota mais aguda).
  }
}

\newglossaryentry{csim}{
  name=CSIM,
  description={
    Função de similaridade de contorno. Mede o grau de similaridade
    entre dois \gls{cseg} de mesma cardinalidade.
  }
}



\graphicspath{{figs-out/}}

\doublespacing

% usar para termos estrangeiros
\newcommand{\eng}[1]{\textit{#1}}

\newcommand{\contorno}[1]{$\langle #1 \rangle$}

% usar para citação integral indentada com tradução
\newcommand{\citacaoindt}[4]{
  \begin{quote}
    \normalsize
%    \selectlanguage{english}
    {#2}\footnote{
      \selectlanguage{brazil}
      ``{#1}''.
    }.
    \selectlanguage{brazil}
    \cite[#3]{#4}.
  \end{quote}
}

\title{Nome da peça: Aplicações da Teoria dos Contornos na composição}
\author{Marcos da Silva Sampaio}

\begin{document}

\maketitle
\tableofcontents
\listoftables
\listoffigures

\chapter{Introdução}
\label{cha:introducao}

A idéia de preservação de contorno é encontrada em diferentes
situações musicais, como nas respostas reais e tonais das fugas
\cite[p. 29]{morris87:composition}.

Teóricos musicais reconhecem que ouvintes têm maior acuidade na
percepção de semelhança de contornos do que na semelhança de
alturas. Por isso novas teorias para comparação de contornos são
necessárias à área de Análise Musical
\cite[p. 226]{marvin.ea87:relating}.

\chapter{Sobre contornos}
\label{cha:sobre-contornos}

%% brainstorm

% análise: comparação e equivalência de materiais.
% análise: objetivos diferentes.
% redução: tipologia de adams e algoritmo de morris
% equivalência a partir de matriz de comparação
% cognição: dowling
% terminologias diferentes: friedmann
% problemas com teorias: clifford
% estruturação da composição a partir de contornos
% exemplos com análise de peças?
% contornos: coerência, schoenberg

Teorias de contornos foram desenvolvidas primariamente como técnicas
analíticas para composições atonais que podem não ter características
musicais usadas para demonstrar coerência em composições tonais
\cite[p. 1]{beard03:contour}.

Contorno é um conjunto ordenado de elementos distintos, com ou sem
repetição, numerados de forma ascendente
\cite[p. 206]{morris93:directions}. Por definição contornos musicais
são ordenados \cite[p. 228]{marvin.ea87:relating}.

Um contorno pode ser interpretado como registro, dinâmica ou densidade
de acordes no tempo \cite[p. 206]{morris93:directions}
\cite[p. 22]{clifford95:contour}. Representações do contorno
\contorno{4\:3\:5\:6} podem ser vistas na figura
\ref{fig:non-melodic-contours}.

\begin{figure}
  \centering
  \subfigure[alturas no tempo]{
    \includegraphics[scale=1]{pitches-in-time}
    \label{fig:pitches-in-time}}
  \qquad
  \subfigure[densidade de acordes no tempo]{
    \includegraphics[scale=1]{chord-densities-in-time}
    \label{fig:chord-densities-in-time}}
  \qquad
  \subfigure[dinâmicas no tempo]{
    \includegraphics[scale=1]{dynamics-in-time}
    \label{fig:dynamics-in-time}}

  \caption{Contornos \contorno{4\:3\:5\:6} não melódicos}
  \label{fig:non-melodic-contours}
\end{figure}

Adams propôs uma tipologia de contornos melódicos para classificação
de melodias indígenas americanas. Esta tipologia está baseada na
redução do contorno de cada melodia a quatro pontos --- nota inicial
(I), nota final (F), nota mais aguda (H) e nota mais grave (L) --- e
na classificação da inclinação entre essas notas. A inclinação entre
as notas pode ser ascendente, descendente ou nula, e repetição de
notas são admitidas. Adams propôs 15 tipos de contornos melódicos,
como se vê na figura \ref{fig:adams-typology}. A inclinação (ou
\eng{slope}) entre a nota inicial e final é indicada por $S1$ ($I >
F$), $S2$ ($I = F$) e $S3$ ($I < F$). A mudança de direção (ou
\eng{deviation}) é indicada por $Dø$ (sem mudança de direção), $D1$
(se H ou L são diferentes de I ou F) e $D2$ (se H e L são diferentes
de I e F). A ordem entre as mudanças de direção, chamada
\eng{reciprocal}, é indicada por $R1$ (H antes de L) e $R2$ (L antes
de H) \cite{adams76:melodic}.

\begin{figure}
  \centering
  \includegraphics[scale=.6]{adams-typology}
  \caption{Tipologia de contornos de Charles Adams
    \cite{adams76:melodic}}
  \label{fig:adams-typology}
\end{figure}

A terminologia utilizada em teorias de contornos melódicos não é
uniforme. Cada autor utiliza termos diferentes para ferramentas
semelhantes \cite{friedmann87:response}. A tabela
\ref{tab:compara-ferramentas} tem uma comparação dos termos usados por
Friedmann, e Morris, Marvin e Laprade.

\begin{table}
  \centering
  \begin{tabular}{l|l}
    Friedmann & Marvin e Laprade \\ \hline
    \gls{cas}  & $INT_1$ \\
  \end{tabular}
  \caption{Quadro comparativo de ferramentas de análise de contornos}
  \label{tab:compara-ferramentas}
\end{table}

Morris define os conceitos de espaço de alturas (\eng{\gls{p-space}})
e espaço de contorno (\eng{\gls{c-space}})
\cite{morris87:composition}. Trata-se de uma abstração. No primeiro os
elementos a serem considerados são as notas e no segundo são os
elementos são os registros dessas notas. Beard confunde a idéia de
espaço do contorno com a definição de classe de contorno
\cite[p. 11]{beard03:contour}.

A Matriz de Comparação (\eng{\gls{com-matrix}}) representa o total de
comparações de um contorno. É obtida através da comparação de todos os
pares ordenados de um contorno. Em uma matriz $E$ de um contorno $P$,
a posição $E (x,y)$ contém $Com (Px,Py)$. A matriz exibe uma simetria
de sinais invertidos em torno da diagonal principal, esta preenchida
apenas por zeros. Um exemplo desta matriz referente ao contorno
$\langle 4\:3\:5\:6 \rangle$ pode ser visto na tabela
\ref{tab:matriz-4356} \cite[p. 28]{morris87:composition}.

\begin{table}
  \centering
  \begin{tabular}{r|cccc}
    & $4$ & $3$ & $5$ & $6$ \\
    \hline
    $4$ & $0$ & $-$ & $+$ & $+$ \\
    $3$ & $+$ & $0$ & $+$ & $+$ \\
    $5$ & $-$ & $-$ & $0$ & $+$ \\
    $6$ & $-$ & $-$ & $-$ & $0$ \\
  \end{tabular}
  \caption{Matriz de comparação do contorno $\langle 4\:3\:5\:6 \rangle$}
\label{tab:matriz-4356}
\end{table}

Cada diagonal à direita da diagonal zero recebe o nome $INT_n$, onde
$n$ representa a diferença de posição entre dois elementos. Dessa
forma $INT_1$ representa as diferenças de altura entre \eng{\gls{cps}}
adjacentes, $INT_2$ representa as diferenças de altura entre
\eng{c-pitches} separados não adjacentes separados por um elemento e
assim por diante \cite[p. 231]{marvin.ea87:relating}. Dado um contorno
$\langle 4\:3\:5\:6 \rangle$, $INT_1 = \langle - + +\rangle$
representa as diferenças entre $4$ e $3$, $3$ e $5$, e $5$ e
$6$. $INT_2 = \langle + + \rangle$ representa as diferenças entre $4$
e $5$, e $3$ e $6$.

Esta matriz permite a verificação de duas classes de equivalência de
contornos. A primeira é formada por todos os \gls{cseg} que dividem
uma mesma \eng{\gls{com-matrix}}, e a segunda ---
\eng{\gls{csegclass}} --- formada por todos os \gls{cseg} relacionados
por operações de identidade, translação, retrogradação, inversão e
retrogradação da inversão.

A operação de inversão é obtida pela subtração de todos os
\eng{\gls{cps}} por $(n-1)$. Para um dado \eng{\gls{cseg}} $P$, a
operação de identidade é identificada como $IP$. A retrogradação de um
\eng{\gls{cseg}} $P$ (representado por $RP$), bem como de sua inversão
($IP$) são obtidas colocando os \eng{\gls{cps}} em ordem reversa
\cite[p. 231]{marvin.ea87:relating}.

A forma prima de um \eng{\gls{cseg}} é obtida operando-se de um
algoritmo de três etapas:
\begin{enumerate}
\item realiza-se a translação do \eng{\gls{cseg}} caso seja necessário;
\item se a substração de $(n-1)$ pelo último \eng{\gls{cps}} é menor
  que o primeiro \eng{\gls{cps}}, inverte-se a \eng{\gls{cseg}};
\item se o último \eng{\gls{cps}} é menor que o primeiro
  \eng{\gls{cps}} retrograda-se a \eng{\gls{cseg}}.
\end{enumerate}

Michael Friedmann procurou desenvolver uma metodologia para estudo
sistemático de contornos \cite{friedmann85:methodology}. Para isso
criou duas ferramentas principais:

\begin{enumerate}
\item \eng{\gls{cas}} (CAS) e
\item \eng{\gls{cc}} (CC).
\end{enumerate}

Há mais seis ferramentas derivadas dessas duas principais:

\begin{enumerate}
\item \eng{\gls{casv}} (CASV),
\item \eng{\gls{ci}} (CI),
\item \eng{\gls{cis}} (CIS),
\item \eng{\gls{cia}} (CIA),
\item \eng{\gls{ccvi}} (CCVI), e
\item \eng{\gls{ccvii}} (CCVII).
\end{enumerate}

A série de classe de contornos adjacentes é um subconjunto da matriz
de comparação representada pela diagonal superior à diagonal zero.

\section{Percepção de contornos}
\label{sec:perc-de-cont}

Contorno melódico é uma importante característica musical no
reconhecimento de melodias familiares
\cite[p. 136]{dowling.ea86:music}. Esta afirmação é endossada por
experimentos realizados por White \cite{white60:recognition}, e
Dowling e Fujitani \cite{dowling.ea71:contour} nos quais ouvintes
foram submetidos ao reconhecimento de versões de canções familiares as
quais intervalos melódicos e/ou ritmo eram modificados e o contorno
preservado.

De acordo com Friedmann as implicações do estudo de contornos na
música do século XX são mais significativas para os ouvintes do que
para os compositores. Isto porque a percepção de contornos é mais
geral do que a percepção de altura e dos outros elementos do universo
da teoria atonal \cite[p. 224]{friedmann85:methodology}.

A idéia de Hindemith sobre percepção de contornos é semelhante à dos
demais autores, embora o contexto em que se aplique seja o do estudo
da harmonia. De acordo com ele é mais fácil lembrar de sucessão
rítmica e de curvas de uma linha melódica do que de diferenças em
tensão entre harmonias \cite[p. 175]{hindemith41:craft}.

\section{Contorno como determinante composicional}
\label{sec:cont-como-determ}

Clifford afirma que

\citacaoindt{contour, in absence of other pervasive systems of pitch
  organization, represents a structural factor equal in significance
  to pitch or set class relations}{contorno, na falta de outros
  sistemas ocupantes de organização de altura, representa um fator
  estrutural igual em significado a relações de alturas ou de classes
  de conjuntos}{p. 157}{clifford95:contour}

No nível melódico relações de contorno podem associar segmentos de
contorno distintos entre dois ou mais segmentos. Neste nível contorno
pode contribuir com um processo de transformação melódica
\cite[p. 159]{clifford95:contour}.

\chapter{Ferramentas}
\label{cha:ferramentas}

\chapter{Análise da composição}
\label{cha:anal-da-comp}

\chapter{Conclusão}
\label{cha:conclusao}

%%% bibliografia
%% teoria dos contornos

\bibliographystyle{plainurl-br}
\bibliography{melodic-contour,music-perception,composition}

\glsaddall{cas,cc,ci,casv,ci,cia,ccvi,ccvii} 
\glsaddall{COM-Matrix,cps,c-space,cseg,csegclass,csubseg}

\printglossary[title=Glossário]

\end{document}
