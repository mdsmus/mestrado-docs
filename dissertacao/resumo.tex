\chapter*{Resumo}
\label{cha:resumo}

%% o que são contornos?

Contorno pode ser entendido como o perfil de um objeto, pode ser
bidimensional, e

%% por que são importantes

Contornos são importantes porque assim como conjuntos de notas e
motivos podem ajudar a dar coerência a uma obra musical.

%% o que fiz

Neste trabalho apresento a obra musical para quinteto de sopros
\obra{}, composta com base em combinações de operações de contornos
associadas a parâmetros como altura, andamento, densidade, e
textura.

%% por que fiz

Verifiquei que teorias de contornos são utilizadas em áreas como
Percepção e Análise Musical, no entanto que o uso sistemático de
contornos para geração de material composicional é assunto carente de
literatura.

Dessa forma este estudo eleva o estado de arte de teorias de contornos
e contribui com novas ferramentas para a área de Composição.

%% como fiz

Para este trabalho fiz uma revisão de literatura de teorias de
contornos, um mapeamento de contornos para elementos musicais, compus
estudos para experimentação de possibilidades com contornos,
desenvolvi o Goiaba, um software para auxiliar no processamento de
contornos para composição, e por fim compus a obra \obra{}.

%% o que concluo

Concluo que contornos podem ser usados de forma sistemática na
composição musical, mas que ainda é necessário um maior aprofundamento
no assunto. Dessa forma este aprofundamento e a continuidade no
desenvolvimento do Goiaba são possíveis atividades futuras decorrentes
deste trabalho.
