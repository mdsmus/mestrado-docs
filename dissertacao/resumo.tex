\chapter*{Resumo}
\label{cha:resumo}

%% o que são contornos?
Contornos podem ser entendidos como perfis ou formatos de objetos. Em
Música contornos podem mapear um parâmetro em função de outro, como
altura em função do tempo ou densidade em função da
amplitude. Contornos são importantes porque, assim como conjuntos de
notas e motivos, podem ajudar a dar coerência a uma obra musical.
%% por que fiz
Teorias de contornos têm sido utilizadas em áreas como Percepção e
Análise Musical, no entanto o uso sistemático de contornos para
geração de material composicional é assunto ainda carente de
literatura.

%% o que fiz
Neste trabalho apresento a obra musical para quinteto de sopros
\obra{}, composta com base em combinações de operações de contornos
associadas a parâmetros como altura, andamento, densidade, e textura.
%% como fiz
Para este trabalho fiz uma revisão de literatura de teorias de
contornos, um mapeamento de contornos para elementos musicais, compus
estudos para experimentação de possibilidades com contornos,
desenvolvi o Goiaba, um software para auxiliar no processamento de
contornos para composição, e por fim compus a obra \obra{}.

%% resultado
Este trabalho ajuda a elevar o estado de arte de teorias de contornos e
contribui com novas ferramentas para a área de Composição.
%% o que concluo
Concluo que contornos podem ser usados de forma sistemática na
composição musical, mas que ainda é necessário um maior aprofundamento
no assunto. Dessa forma este aprofundamento e a continuidade no
desenvolvimento do Goiaba são possíveis atividades futuras decorrentes
deste trabalho.
