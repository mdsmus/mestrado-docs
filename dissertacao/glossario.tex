\makeglossaries
\glossarystyle{altlistgroup}

\newglossaryentry{cas}{
  name=Contour Adjacency Series,
  description={
    Série de elementos adjacentes de um contorno. É uma série de sinais
    '+' e '-' referentes às inclinações positivas e negativas entre notas
    adjascentes de um contorno. Contornos adjacentes de mesma altura são
    desconsiderados, assim como não há uma indicação de repetição de
    altura em larga escala. Por exemplo, para um contorno \texttt{d c f
      d}, CAS = \texttt{$<$-,+,-$>$}; para um contorno \texttt{d c c e}, CAS =
    \texttt{$<$-,+$>$}.
    A principal utilidade da CAS está na comparação de equivalência com
    retrógrado ou rotação de ordem dos elementos do contorno.
  }
} 

\newglossaryentry{cc}{
  name=Contour Class,
  description={
    Classe de contornos. É uma série ordenada de números que representa a
    relação de alturas entre todas as notas de um contorno, na qual a nota
    mais grave é indicada por 0 e a nota mais aguda por n-1, onde n=total
    de notas. Por exemplo, para um contorno \texttt{d c f e}, CC = \texttt{$<$1 0
      3 2$>$}.
  }
}

\newglossaryentry{casv}{
  name=Contour Adjacency Series Vector,
  description={
    Vetor de classe de contornos adjacentes. É um vetor de dois dígitos
    com a soma das inclinações positivas e negativas de um contorno. Por
    exemplo, para um contorno \texttt{d c f e} CASV = \texttt{$<$1,2$>$}.
    Sua principal utilidade é estabelecer uma classe de equivalência entre
    rotações, retrógrados e outras permutações de ordem de elementos.
  }
}

\newglossaryentry{ci}{
  name=Contour Interval,
  description={
    Intervalo de contorno. É a distância entre dois elementos de uma
    classe de contorno (CC). Por exemplo, em CC $<$1 0 3 2$>$, a CI de \texttt{1-0} é
    \texttt{-1}, a CI de \texttt{0-3} é \texttt{+3}.
  }
}

\newglossaryentry{cis}{
  name=Contour Interval Succession,
  description={
    Sucessão de intervalos de contorno é uma série de números que
    representam os intervalos de contorno (CI) de uma classe de contorno
    (CC). Por exemplo, em CC \texttt{$<$1 0 3 2$>$}, CIS = \texttt{$<$1 3 -1$>$}.
  }
}

\newglossaryentry{cia}{
  name=Contour Interval Array,
  description={
    Vetor de intervalos de contorno. É uma série de números que representa
    a multiplicidade de intervalos de contornos (CI) de uma classe de
    contorno (CC). Esta série é dividida em duas e cada número representa
    o número de CI's de valor \texttt{1}, \texttt{2}, \texttt{3}, etc, a
    depender do tamanho do contorno. A primeira parte da série representa
    CI's positivos e a segunda representa CI's negativos. Por exemplo,
    para CC = \texttt{$<$1 0 3 2$>$}, CIA = \texttt{$<$1 2 1$>$,$<$2 0
      0$>$}. Esta CC tem um intervalo de valor \texttt{+1}, dois de valor
    \texttt{+2}, um de valor \texttt{+3}, dois de valor \texttt{-1} e
    nenhum de valores \texttt{-2} e \texttt{-3}.
  }
}

\newglossaryentry{ccvi}{
  name=Contour Class Vector I,
  description={
    Vetor de classe de contorno I. É um par de números que representa o
    grau de ascendência e descendência de uma classe de contorno (CC). É
    obtido através da soma da multiplicação do número de ocorrências de
    cada intervalo de contorno (CI) pelo próprio valor do intervalo. Por
    exemplo, em um CIA \texttt{$<$1 2 1$>$,$<$2 0 0$>$} CCVI =
    \texttt{$<$8 2$>$}.
  }
}

\newglossaryentry{ccvii}{
  name=Contour Class Vector II,
  description={
    Vetor de classe de contorno II. É um par de números que representa o
    grau de ascendência e descendência de uma classe de contorno (CC). É
    obtido através da soma dos valores positivos e negativos das
    ocorrências de intervalos de contorno (CI). Por exemplo, em um CIA
    \texttt{$<$1 2 1$>$,$<$2 0 0$>$}, CCVII = \texttt{$<$8 2$>$}
  }
}
%%% glossário de marvin.ea87:relating

\newglossaryentry{com-matrix}{
  name=COM-Matrix,
  description={
    Matriz de comparação. Uma matriz bidimensional que mostra os
    resultados da função de comparação $COM(a,b)$ para qualquer altura
    de contorno \eng{c-pitch} no espaço de contorno \eng{c-space}. Se
    $b > a$ a função retorna ``$+$''. Se $b = a$ a função retorna
    $0$. Se $b < a$ a função retorna ``$-$''.
  }
}

\newglossaryentry{cps}{
  name=c-pitches,
  description={
    Altura de contorno. Elementos no espaço de contorno numerados de
    forma ordenada do mais grave para o mais agudo, de $0$ a $(n -
    1)$, em que n representa o número de elementos.
  } 
}

\newglossaryentry{c-space}{
  name=Contour Space,
  description={
    Espaço de contorno. Um tipo de espaço musical consistindo de
    elementos organizados do grave para o agudo desconsiderando os
    intervalos exatos entre os elementos.
  }
}

\newglossaryentry{cseg}{
  name=c-segment,
  description={
    Segmento de contorno. Um conjunto ordenado de alturas de contorno
    no espaço de contorno \gls{c-space}.
  }
}

\newglossaryentry{csegclass}{
  name=c-space segment class,
  description={
    Classe de segmentos do espaço de contorno. Uma classe de
    equivalência feita de todos os segmentos de contorno relacionados
    por identidade, translação, retrogradação, inversão e
    retrogradação da inversão.
  }
}

\newglossaryentry{csubseg}{
  name=c-subsegment,
  description={
    Subsegmento de contorno. Qualquer subgrupo de um dado segmento de
    contorno. Pode ser compreendido de alturas de contornos adjacentes
    ou não do segmento de contorno original.
  }
}
