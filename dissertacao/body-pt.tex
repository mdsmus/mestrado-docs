\chapter{Introdução}
\label{cha:introducao}

A idéia de preservação de contorno e variação de intervalos entre
notas é encontrada em diferentes situações musicais. Há adequação de
notas à tonalidade em respostas tonais de fugas, em mudanças de modo
em peças do tipo tema e variações, e em \eng{leitmotifs} e idéias
fixas, citando apenas exemplos óbvios
\cite[p. 29]{morris87:composition}. Outras obras têm motivos cujos
intervalos são pouco a pouco expandidos ou contraídos, como por
exemplo ocorre no início da \opus{Música para Cordas, Percussão e
  Celesta}, de Béla Bartók.

Teóricos musicais reconhecem que ouvintes têm maior acuidade na
percepção de semelhança de contornos do que na semelhança de
alturas. Por isso novas teorias para comparação de contornos se
tornaram necessárias à área da Análise Musical
\cite[p. 226]{marvin.ea87:relating}.

\chapter{Sobre contornos}
\label{cha:sobre-contornos}

%% brainstorm

% análise: comparação e equivalência de materiais.
% análise: objetivos diferentes.
% redução: tipologia de adams e algoritmo de morris
% equivalência a partir de matriz de comparação
% cognição: dowling
% terminologias diferentes: friedmann
% problemas com teorias: clifford
% estruturação da composição a partir de contornos
% exemplos com análise de peças?
% contornos: coerência, schoenberg

Teorias de contornos foram desenvolvidas primariamente como técnicas
analíticas para composições atonais que podem não ter características
musicais usadas para demonstrar coerência em composições tonais
\cite[p. 1]{beard03:contour}.

Embora teorias de contorno não tenham surgido para análise de obras
tonais, a análise a partir da perspectiva dos contornos já se mostrou
eficiente também neste tipo de obra. Assim, além de obras de Schönberg
\cite{friedmann85:methodology} e Webern \cite{clifford95:contour},
sonatas para piano de Mozart \cite{beard03:contour} já foram
analisadas sob a ótica das teorias de contornos.

Contorno é um conjunto ordenado de elementos distintos, com ou sem
repetição, numerados de forma ascendente
\cite[p. 206]{morris93:directions}. Por definição contornos musicais
são ordenados \cite[p. 228]{marvin.ea87:relating}.

Um contorno pode ser interpretado como registro, dinâmica ou densidade
de acordes no tempo \cite[p. 206]{morris93:directions}
\cite[p. 22]{clifford95:contour}. Podemos ver representações do
contorno \contorno{4\:3\:5\:6} na figura
\ref{fig:non-melodic-contours}.

\begin{figure}
  \centering
  \subfloat[alturas no tempo]{
    \includegraphics[scale=1]{pitches-in-time}
    \label{fig:pitches-in-time}}
  \subfloat[densidade de acordes no tempo]{
    \includegraphics[scale=1]{chord-densities-in-time}
    \label{fig:chord-densities-in-time}}

  \subfloat[dinâmicas no tempo]{
    \includegraphics[scale=1]{dynamics-in-time}
    \label{fig:dynamics-in-time}}
  \caption{Contornos \contorno{4\:3\:5\:6} não melódicos}
  \label{fig:non-melodic-contours}
\end{figure}

Adams propôs uma tipologia de contornos melódicos para classificação
de melodias indígenas americanas. Esta tipologia está baseada na
redução do contorno de cada melodia a quatro pontos --- nota inicial
(I), nota final (F), nota mais aguda (H) e nota mais grave (L) --- e
na classificação da inclinação entre essas notas. A inclinação entre
as notas pode ser ascendente, descendente ou nula, e repetição de
notas são admitidas. Adams propôs 15 tipos de contornos melódicos,
como se vê na figura \ref{fig:adams-typology}. A inclinação (ou
\eng{slope}) entre a nota inicial e final é indicada por $S1$ ($I >
F$), $S2$ ($I = F$) e $S3$ ($I < F$). A mudança de direção (ou
\eng{deviation}) é indicada por $Dø$ (sem mudança de direção), $D1$
(se H ou L são diferentes de I ou F) e $D2$ (se H e L são diferentes
de I e F). A ordem entre as mudanças de direção, chamada
\eng{reciprocal}, é indicada por $R1$ (H antes de L) e $R2$ (L antes
de H) \cite{adams76:melodic}.

\begin{figure}
  \centering
  \includegraphics[scale=.6]{adams-typology}
  \caption{Tipologia de contornos de Charles Adams}
  \label{fig:adams-typology}
\end{figure}

A terminologia utilizada em teorias de contornos melódicos não é
uniforme. Cada autor utiliza termos diferentes para operações
semelhantes \cite{friedmann87:response}. A tabela
\ref{tab:compara-ferramentas} tem uma comparação dos termos usados por
Friedmann, e Morris, Marvin e Laprade.

\begin{table}
  \centering
  \begin{tabular}{l|l}
    Friedmann & Marvin e Laprade \\ \hline
    \eg{cas}  & $INT_1$ \\
  \end{tabular}
  \caption{Quadro comparativo de ferramentas de análise de contornos}
  \label{tab:compara-ferramentas}
\end{table}

Morris define os conceitos de espaço de alturas (\eg{p-space}) e
espaço de contorno (\eg{c-space}) \cite{morris87:composition}. Em
\eg{p-space} os elementos a serem considerados são as notas e em
\eg{c-space} os elementos são os registros dessas notas. Beard
confunde a idéia de espaço do contorno com a definição de classe de
contorno \cite[p. 11]{beard03:contour}.

A Matriz de Comparação (\eg{com-matrix}) representa o total de
comparações de um contorno. É obtida através da comparação de todos os
pares ordenados de um contorno. Em uma matriz $E$ de um contorno $P$,
a posição $E (x,y)$ contém $Com (Px,Py)$. A matriz exibe uma simetria
de sinais invertidos em torno da diagonal principal, esta preenchida
apenas por zeros. Um exemplo desta matriz referente ao contorno
$\langle 4\:3\:5\:6 \rangle$ pode ser visto na tabela
\ref{fig:matriz-4356} \cite[p. 28]{morris87:composition}.

\begin{figure}
  \centering
  \begin{tabular}{r|cccc}
    & $4$ & $3$ & $5$ & $6$ \\
    \hline
    $4$ & $0$ & $-$ & $+$ & $+$ \\
    $3$ & $+$ & $0$ & $+$ & $+$ \\
    $5$ & $-$ & $-$ & $0$ & $+$ \\
    $6$ & $-$ & $-$ & $-$ & $0$ \\
  \end{tabular}
  \caption{Matriz de comparação do contorno $\langle 4\:3\:5\:6 \rangle$}
  \label{fig:matriz-4356}
\end{figure}

Cada diagonal à direita da diagonal zero recebe o nome $INT_n$, onde
$n$ representa a diferença de posição entre dois elementos. Dessa
forma $INT_1$ representa as diferenças de altura entre \eg{cps}
adjacentes, $INT_2$ representa as diferenças de altura entre
\eng{c-pitches} separados não adjacentes separados por um elemento e
assim por diante \cite[p. 231]{marvin.ea87:relating}. Dado um contorno
$\langle 4\:3\:5\:6 \rangle$, $INT_1 = \langle - + +\rangle$
representa as diferenças entre $4$ e $3$, $3$ e $5$, e $5$ e
$6$. $INT_2 = \langle + + \rangle$ representa as diferenças entre $4$
e $5$, e $3$ e $6$.

Esta matriz permite a verificação de duas classes de equivalência de
contornos. A primeira é formada por todos os \eg{cseg} que dividem
uma mesma \eg{com-matrix}, e a segunda ---
\eg{csegclass} --- formada por todos os \eg{cseg} relacionados
por operações de identidade, translação, retrogradação, inversão e
retrogradação da inversão.

A operação de inversão é obtida pela subtração de todos os
\eg{cps} por $(n-1)$. Para um dado \eg{cseg} $P$, a
operação de identidade é identificada como $IP$. A retrogradação de um
\eg{cseg} $P$ (representado por $RP$), bem como de sua inversão
($IP$) são obtidas colocando os \eg{cps} em ordem reversa
\cite[p. 231]{marvin.ea87:relating}.

A forma prima de um \eg{cseg} é obtida operando-se de um
algoritmo de três etapas:
\begin{enumerate}
\item realiza-se a translação do \eg{cseg} caso seja necessário;
\item se a subtração de $(n-1)$ pelo último \eg{cps} é menor
  que o primeiro \eg{cps}, inverte-se a \eg{cseg};
\item se o último \eg{cps} é menor que o primeiro
  \eg{cps} retrograda-se a \eg{cseg}.
\end{enumerate}

Michael Friedmann procurou desenvolver uma metodologia para estudo
sistemático de contornos \cite{friedmann85:methodology}. Para isso
criou duas ferramentas principais:

\begin{enumerate}
\item \eg{cas} (CAS) e
\item \eg{cc} (CC).
\end{enumerate}

Há mais seis ferramentas derivadas dessas duas principais:

\begin{enumerate}
\item \eg{casv} (CASV),
\item \eg{ci} (CI),
\item \eg{cis} (CIS),
\item \eg{cia} (CIA),
\item \eg{ccvi} (CCVI), e
\item \eg{ccvii} (CCVII).
\end{enumerate}

A série de classe de contornos adjacentes é um subconjunto da matriz
de comparação representada pela diagonal superior à diagonal zero.

\section{Percepção de contornos}
\label{sec:perc-de-cont}

Contorno melódico é uma importante característica musical no
reconhecimento de melodias familiares
\cite[p. 136]{dowling.ea86:music}. Esta afirmação é endossada por
experimentos realizados por White \cite{white60:recognition}, e
Dowling e Fujitani \cite{dowling.ea71:contour} nos quais ouvintes
foram submetidos ao reconhecimento de versões de canções familiares as
quais intervalos melódicos e/ou ritmo eram modificados e o contorno
preservado.

Implicações do estudo de contornos na música do século XX são mais
significativas para os ouvintes do que para os compositores. Isto
porque a percepção de contornos é mais geral do que a percepção de
altura e dos outros elementos do universo da teoria atonal
\cite[p. 224]{friedmann85:methodology}.

A idéia de Hindemith sobre percepção de contornos é semelhante à dos
demais autores, embora o contexto em que se aplique seja o do estudo
da harmonia. De acordo com ele é mais fácil lembrar de sucessão
rítmica e de curvas de uma linha melódica do que de diferenças em
tensão entre harmonias \cite[p. 175]{hindemith41:craft}.

\section{Contorno como determinante composicional}
\label{sec:cont-como-determ}

Clifford afirma que

%%% quando citamos literalmente, é bom fazer algum comentario

\citacaolonga{contour, in absence of other pervasive systems of pitch
  organization, represents a structural factor equal in significance
  to pitch or set class relations}{contorno, na falta de outros
  sistemas ocupantes de organização de altura, representa um fator
  estrutural igual em significado a relações de alturas ou de classes
  de conjuntos}{p. 157}{clifford95:contour}

No nível melódico relações de contorno podem associar segmentos de
contorno distintos entre dois ou mais segmentos. Neste nível contorno
pode contribuir com um processo de transformação melódica
\cite[p. 159]{clifford95:contour}.

\chapter{Ferramentas}
\label{cha:ferramentas}

\chapter{Análise da composição}
\label{cha:anal-da-comp}

%% falar de contornos, motivos, forma, altura, timbre, gestual

A obra é baseada em um motivo principal (vide figura
\ref{fig:motivo-principal}) e em derivações do seu contorno. A classe
deste contorno é representada por P(5 3 4 1 2 0).

\begin{figure}
  \centering
  \includegraphics{motivo-principal}
  \caption{Motivo principal}
  \label{fig:motivo-principal}
\end{figure}


Esta obra contém sete seções, conforme indicado na tabela
\ref{tab:secoes-obra}.

\begin{table}
  \centering
  \begin{tabular}{r|ccccccc}
    Seção & 1 & 2 & 3 & 4 & 5 & 6 & 7 \\
    \hline
    Início (letra ensaio) & - & E & H & L & O & R & U \\
    Início (comp.) & 1 & 37 & 57 & 105 & 134 & 173 & 215 \\
    Final (comp.) & 36 & 56 & 104 & 133 & 172 & 214 & 244 \\
    Duração aprox. (s) & 132 & 91 & 108 & 56 & 87 & 23 & 16\\
    Andamento M.M & 82 & 66 & 120 & 120 & 108 & 112 & 112 \\
  \end{tabular}
  \caption{Seções formais da obra}
  \label{tab:secoes-obra}
\end{table}


Contornos estão associados a outros parâmetros musicais como
andamento, densidade e textura.

Os andamentos utilizados na peça---82, 66, 120, 108 e
112---representam o contorno A(1 0 4 2 3). Este contorno é um
subconjunto de 5 elementos do contorno principal utilizado, P(5 3 4 1
2 0).

Na subseção AA a densidade é representada pelo contorno D(0 2 1 4
3). Esta seção é iniciada com um solo de fagote (0) seguido de um trio
com fagote, clarinete e oboé (2). Ocorre então um duo entre clarinete
e flauta (1), um \eng{tutti} (4) e finalmente um quarteto sem o fagote
(3). Este contorno D é um retrógrado de um subconjunto de 5 elementos
do contorno principal P.

As texturas presentes na peça podem ser divididas em dois grandes
grupos: de texturas homofônicas, que engloba texturas corais e de
melodia acompanhada; e de texturas polifônicas, que engloba textura
contrapontística e textura complexa. Estes grupos são apresentados
nesta ordem:
polifonia-homofonia-polifonia-homofonia-polifonia-homofonia. Considerando
que uma textura polifônica é mais complexa que uma textura homofônica,
a alternância entre estas texturas delineia um contorno de $INT_1$ (-
+ - + -), derivado do contorno principal P(5 3 4 1 2 0).

\chapter{Conclusão}
\label{cha:conclusao}
