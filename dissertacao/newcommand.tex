% usar para termos estrangeiros
\newcommand{\eng}[1]{\textit{#1}}
% \newcommand{\eg}[1]{\textit{\gls{#1}}}
\newcommand{\eg}[1]{\textit{#1}}

\newcommand{\opus}[1]{\textit{#1}}
\newcommand{\tr}[1]{\textit{#1}}

\newcommand{\contorno}[1]{$\langle #1 \rangle$}
\newcommand{\contpr}{P(5 3 4 1 2 0)}
\newcommand{\obra}{\textit{Peça ainda sem título}}
\newcommand{\goiaba}{\opus{Goiaba}}

% usar para citação integral indentada com tradução
\newcommand{\citacaolonga}[4]{
  \begin{quote}
    \normalsize
%    \selectlanguage{english}
    {#1}\footnote{
      \selectlanguage{brazil}
      ``{#2}''.
    }.
    \selectlanguage{brazil}
    \cite[#3]{#4}.
  \end{quote}
}

\newcounter{notecounter}

\newcommand{\note}[1]{
  \addtocounter{notecounter}{1}
  \textcolor{red}{[note \arabic{notecounter}: #1]}
}

