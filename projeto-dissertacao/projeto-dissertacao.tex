\documentclass{article}
%\usepackage{mdschicago}
\usepackage{ifthen}
\usepackage[utf8,utf8x]{inputenc}
\usepackage[a4paper,top=2cm,bottom=2cm,left=2cm,right=2cm]{geometry}
\usepackage[brazil]{babel}
\usepackage{setspace}
\usepackage{graphicx}
\usepackage{url}

% usar para termos estrangeiros
\newcommand{\eng}[1]{\textit{#1}}

% usar para nomes de obras
\newcommand{\opus}[1]{\textit{#1}}

% usar para nomes de termos
\newcommand{\termo}[1]{\textit{#1}}

\begin{document}

\title{Projeto de Dissertação}
\author{Marcos da Silva Sampaio}
\maketitle

\thispagestyle{empty}

\setlength{\parindent}{0cm}

\section{Introdução}
\label{sec:introducao}

Contorno pode ser definido como o perfil, desenho ou formato de um
objeto. Pode ser bidimensional e associar altura a comprimento,
largura ou tempo. Em música contornos podem ser associados a altura,
densidade, ritmo, complexidade rítmica, homogeneidade orquestral,
harmônicos, intensidade, etc. Contornos melódicos estão relacionados
com movimento de altura em função do tempo.

%% Contornos melódicos na composição

Em composição o contorno melódico é um elemento que se situa em um
nível médio de abstração entre altura, duração e timbre, e motivos e
frases. Sua presença na música é tão comum quanto a de qualquer outro
elemento citado.

%% Geração de material composicional

Ao estruturar uma obra musical compositores utilizam recursos como
proporções matemáticas, relações entre notas (conjuntos,
tonalidades).

%% Teorias de contornos

Teorias de contornos têm sido desenvolvidas desde os anos 1980 para
uso em análise musical
\cite{friedmann85:_method_discus_contour,friedmann1987rmc,marvin87:_relat_music_contour,polansky92:_possib_impos_melod,quinn97:_fuzzy_exten_theor_contour}.

%% contornos em análise e em composição

Estes estudos sobre contornos melódicos têm sido desenvolvidos para
análise musical. Nenhum dos trabalhos citados, entretanto, aborda o
uso de contornos melódicos para composição.

\section{Objetivos}
\label{sec:objetivos}

O principal objetivo deste projeto é  a composição de uma peça e o seu
memorial.  Esta peça, escrita  para quinteto  de madeiras,  deverá ser
construída com base em material  desenvolvido a partir de operações em
contornos    melódicos   como    transposição,    rotação,   inversão,
retrogradação, aumentação e diminuição, justaposição e superposição.

São objetivos secundários deste trabalho:

\begin{itemize}
\item Desenvolvimento de um programa de computador para processamento
  de operações de contornos melódicos.
%% TODO:
\item Entendimento do mapeamento de contornos para elementos
  musicais/composicionais
\end{itemize}

\section{Justificativa}
\label{sec:justificativa}

%% apenas 1 parágrafo

Teorias de contornos melódicos são utilizadas para análise
composicional, porém não para composição.

\section{Revisão de Literatura}
\label{sec:revis-de-liter}

\cite{friedmann85:_method_discus_contour},
\cite{morris1987cpc}, \cite{marvin87:_relat_music_contour}
e \cite{polansky92:_possib_impos_melod} iniciaram o
desenvolvimento de teorias de contornos musicais expandindo o trabalho
anterior de \cite{seeger1960mml},
\cite{kolinkski65:_struc_melod_movem} e
\cite{adams1976mct}. Mais tarde essas teorias foram expandidas
por \cite{quinn97:_fuzzy_exten_theor_contour} e por
\cite{beard2003cmm}.


\section{Estrutura de tópicos da dissertação}
\label{sec:estrutura-de-topicos}

\begin{enumerate}
\item Introdução
\item Revisão de Literatura `sobre contornos melódicos'
\item Análise da composição
\item Partitura da composição
\item Conclusão
\item Bibliografia
\end{enumerate}

\section{Metodologia}
\label{sec:metodologia}
%% descrever o que será feito

\begin{enumerate}
\item Revisão de literatura sobre contornos melódicos

  Para a realização deverá ser feita uma revisão de literatura sobre
  contornos melódicos.

\item Mapeamento de contornos para elementos musicais

\item Composição de peças didáticas para experimento de possibilidades
  com contornos melódicos.

  Duas peças devem ser compostas para experimentar possibilidades de
  uso de contornos melódicos e operações. A primeira, já concluída,
  contém um único contorno melódico e experimenta possibilidades de
  preenchimento entre os pontos do contorno. A segunda, em andamento,
  testa possibilidades de operações e concatenação. Com esses pequenos
  experimentos é possível aprofundar a prática de composição com
  contornos.

\item Desenvolvimento de um programa de computador para processar
  operações com contornos melódicos.

  Um software que lida com contornos melódicos, operações e
  combinações está sendo desenvolvido. Ele retornará representações
  simbólicas, gráficas e musicais de operações de transposição,
  inversão, retrogradação e rotação de um dado contorno. Além disso
  ele permitirá a combinação de operações e sua concatenação. A
  linguagem é Lisp e a plataforma Unix.

%% Dizer pq é importante
  
\item Composição da peça

  Todo o material composicional a ser utilizado na peça deverá ser
  gerado por operações e concatenação de operações de contornos
  melódicos, e por preenchimento desses contornos.

\item Escrita da dissertação
\end{enumerate}

\section{Resultados Esperados}
\label{sec:resultados-esperados}

Os resultados pretendidos com este trabalho são a obra para quinteto
de madeiras e a dissertação de mestrado, que incluirá a análise da
referida obra.

\section{Cronograma}
\label{sec:cronograma}

\begin{table}[h]
  \centering
  \begin{tabular}{lllllllllllllll}
    \hline
    Tarefa & Nov. & Dez. & Jan. & Fev. & Mar. & Abr. & Mai. & Jun. & Jul. & Ago. & Set. &
    Out. & Nov. & Dez. \\
    \hline
    Revisão de Literatura & X &  &  & & & & & & & & & & & \\
    Experimentos & X &  &  & & & & & & & & & & & \\
    Software & X & X & X & & & & & & & & & & & \\
    Composição da peça & & X & X & X & & & & & & & & & & \\
    Dissertação & & & & & & & & X & X & X & X & X & & \\
    Defesa & & & & & & & & & & & & & & X \\
    \hline


    \hline
  \end{tabular}
%   \caption{}
  \label{tab:cronograma}
\end{table}


\renewcommand{\refname}{Bibliografia}

\nocite{adams1976mct,albrecht2004ect,alves05:_invar,alves2001cuc,batke2004edm,beard2003cmm,Blackburn2000thesis,buteau2005ama,buteau2003tmm,buteau1999mta,buteau2000csm,cambouropoulos2001mca,cambouropoulos2005pea,clifford1995cse,cook87:_techn_compar_analy,discipio2000ajc,dibben01:_motiv_struc_and_percep_of_simil,dowling1994mch,dowling1978sac,dowling1971rim,dowling1986mc,edworthy1985mca,forte83:_motiv_rhyth_contour_alto_rhaps,friedmann1987rmc,friedmann85:_method_discus_contour,ghias1995qhm,Hermann1995,hofmannengl1999vpa,ishiyama1996act,kim2000acb,kolinkski65:_struc_melod_movem,Li2004,lindsay1996ucm,logan2003tet,maidin:gam,marvin1988gtm,marvin87:_relat_music_contour,morris93:_new_direc_theor_analy_music_contour,morris1987cpc,AdamOckelford01012004,parsons1975dta,pauws2002cfo,polansky87mma,polansky92:_possib_impos_melod,prechelt2001imi,quinn2006mcp,quinn97:_fuzzy_exten_theor_contour,raju2002tam,raju2003tqh,ravenscroft2006rcc,music02:_melody,roy2004cbm,schmuckler1999tmm,schubert2006eih,schonberg1967fmc,seeger1960mml,serafine1989idm,sonoda2002bmr,Siu-LanTan04012004,toch1977sfm,toiviainen2002cmm,uitdenbogerd2003pre,weyde2005emr,yi1991tmc,zhu2002smc}

\bibliography{mestrado}
\bibliographystyle{plain}

\end{document}