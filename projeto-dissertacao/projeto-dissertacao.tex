\documentclass{article}
\usepackage{mdschicago}
\usepackage{ifthen}
\usepackage[T1]{fontenc}
\usepackage[a4paper,top=2cm,bottom=2cm,left=2cm,right=2cm]{geometry}
\usepackage[brazil]{babel}
\usepackage{setspace}
\usepackage{graphicx}
\usepackage{url}

% usar para termos estrangeiros
\newcommand{\eng}[1]{\textit{#1}}

% usar para nomes de obras
\newcommand{\opus}[1]{\textit{#1}}

% usar para nomes de termos
\newcommand{\termo}[1]{\textit{#1}}

\begin{document}

\title{Projeto de Dissertação}
\author{Marcos da Silva Sampaio}
\maketitle

\thispagestyle{empty}

\setlength{\parindent}{0cm}

\section{Introdução}
\label{sec:introducao}

\section{Objetivos}
\label{sec:objetivos}

O principal objetivo deste projeto é a composição de uma peça e o seu
memorial. Esta peça, escrita para quinteto de madeiras, deverá ser
construída a partir de operações de transposição, rotação, inversão,
retrogradação, aumentação e diminuição, justaposição e superposição de
contornos melódicos.

São objetivos secundários deste trabalho:

\begin{itemize}
\item Desenvolvimento de um programa de computador para processamento
  de operações de contornos melódicos.
\end{itemize}

\section{Justificativa}
\label{sec:justificativa}

%% apenas 1 parágrafo

Teorias de contornos melódicos são utilizadas para análise
composicional, porém não para composição.

\section{Revisão de Literatura}
\label{sec:revis-de-liter}

\shortciteN{friedmann85:_method_discus_contour},
\shortciteN{morris1987cpc}, \shortciteN{marvin87:_relat_music_contour}
e \shortciteN{polansky92:_possib_impos_melod} iniciaram o
desenvolvimento de teorias de contornos musicais expandindo o trabalho
anterior de \shortciteN{seeger1960mml},
\shortciteN{kolinkski65:_struc_melod_movem} e
\shortciteN{adams1976mct}. Mais tarde essas teorias foram expandidas
por \shortciteN{quinn97:_fuzzy_exten_theor_contour} e por
\shortciteN{beard2003cmm}.


\section{Estrutura de tópicos da dissertação}
\label{sec:estrutura-de-topicos}

\begin{enumerate}
\item Introdução
\item Revisão de Literatura 'sobre contornos melódicos'
\item Análise da composição ainda sem título definido
\item Partitura da composição
\item Conclusão
\item Bibliografia
\end{enumerate}

\section{Cronograma}
\label{sec:cronograma}

\begin{table}[h]
  \centering
  \begin{tabular}{lllllllllllll}
    \hline
    Tarefa & Jan. & Fev. & Mar. & Abr. & Mai. & Jun. & Jul. & Ago. & Set. &
    Out. & Nov. & Dez. \\
    \hline


    \hline
  \end{tabular}
%   \caption{}
  \label{tab:cronograma}
\end{table}


\section{Metodologia}
\label{sec:metodologia}
%% descrever o que será feito

\begin{enumerate}
\item Revisão de literatura sobre contornos melódicos
\item Composição de peças didáticas para experimento de possibilidades
  com contornos melódicos
\item Desenvolvimento de um programa de computador para processar
  operações com contornos melódicos
\item Composição da peça
\end{enumerate}

\renewcommand{\refname}{Bibliografia}

\nocite{adams1976mct,albrecht2004ect,alves05:_invar,alves2001cuc,batke2004edm,beard2003cmm,Blackburn2000thesis,buteau2005ama,buteau2003tmm,buteau1999mta,buteau2000csm,cambouropoulos2001mca,cambouropoulos2005pea,clifford1995cse,cook87:_techn_compar_analy,discipio2000ajc,dibben01:_motiv_struc_and_percep_of_simil,dowling1994mch,dowling1978sac,dowling1971rim,dowling1986mc,edworthy1985mca,forte83:_motiv_rhyth_contour_alto_rhaps,friedmann1987rmc,friedmann85:_method_discus_contour,ghias1995qhm,Hermann1995,hofmannengl1999vpa,ishiyama1996act,kim2000acb,kolinkski65:_struc_melod_movem,Li2004,lindsay1996ucm,logan2003tet,maidin:gam,marvin1988gtm,marvin87:_relat_music_contour,morris93:_new_direc_theor_analy_music_contour,morris1987cpc,AdamOckelford01012004,parsons1975dta,pauws2002cfo,polansky87mma,polansky92:_possib_impos_melod,prechelt2001imi,quinn2006mcp,quinn97:_fuzzy_exten_theor_contour,raju2002tam,raju2003tqh,ravenscroft2006rcc,music02:_melody,roy2004cbm,schmuckler1999tmm,schubert2006eih,schonberg1967fmc,seeger1960mml,serafine1989idm,sonoda2002bmr,Siu-LanTan04012004,toch1977sfm,toiviainen2002cmm,uitdenbogerd2003pre,weyde2005emr,yi1991tmc,zhu2002smc}

\bibliography{mestrado}
\bibliographystyle{mdschicago}

\end{document}