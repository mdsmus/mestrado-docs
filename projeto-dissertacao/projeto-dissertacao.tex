\documentclass{article}
\usepackage{ppgmus}
\usepackage[utf8,utf8x]{inputenc}

\begin{document}

\cabecalho{792}{Pedro Kröger}
\titulo{Projeto de Dissertação}

%% TODO: listar exemplos de contornos, e.g. adams
%%       mostrar exemplos de utilização composicional
%%      (tipo: o que temos feito, operações)
%% enfatizar que terá uma maneira sistematica para trabalhar com
%% contornos
%% "contorno como determinante composicional"

%% falar que assim como tem set, simetria, etc como determinante
%% composicional, contornos podem ser interessantes

\section{Introdução}
\label{sec:introducao}

Contorno pode ser definido como o perfil, desenho ou formato de um
objeto. Pode ser bidimensional e associar altura a comprimento,
largura ou tempo. Em música contornos podem ser associados a altura,
densidade, ritmo, complexidade rítmica, homogeneidade orquestral,
harmônicos, intensidade, etc. Contornos melódicos estão relacionados
com movimento de altura em função do tempo.

%% Contornos melódicos na composição

Em composição o contorno melódico é um elemento de nível médio de
abstração, entre altura, duração e timbre, e motivos e frases. Sua
presença na música é tão comum quanto a de qualquer outro elemento
citado.

%% Geração de material composicional

Ao estruturar uma obra musical compositores utilizam recursos como
proporções matemáticas, relações entre notas (conjuntos, tonalidades).

%% Teorias de contornos

Teorias de contornos têm sido desenvolvidas desde os anos 1980 para
uso em análise musical
\cite{friedmann85:_method_discus_contour,friedmann1987rmc,marvin87:_relat_music_contour,polansky92:_possib_impos_melod,quinn97:_fuzzy_exten_theor_contour}. Algumas
dessas teorias se assemelham à teoria dos conjuntos.

%% contornos em análise e em composição

Estes estudos sobre contornos melódicos têm sido desenvolvidos para
análise musical. Nenhum dos trabalhos citados, entretanto, aborda o
uso de contornos melódicos para composição.

\section{Objetivos}
\label{sec:objetivos}

O principal objetivo deste projeto é  a composição de uma peça e o seu
memorial.  Esta peça, escrita  para quinteto  de madeiras,  deverá ser
construída com base em material  desenvolvido a partir de operações em
contornos    melódicos   como    transposição,    rotação,   inversão,
retrogradação, aumentação e diminuição, justaposição e superposição.

São objetivos secundários deste trabalho:

\begin{itemize}
\item Desenvolvimento de um programa de computador para processamento
  de operações de contornos melódicos.
%% TODO:
\item Entendimento do mapeamento de contornos para elementos
  musicais/composicionais
\end{itemize}

\section{Justificativa}
\label{sec:justificativa}

%% apenas 1 parágrafo

Teorias de contornos melódicos são utilizadas para análise
composicional, porém não para composição.

%% mudar titulo (algo como contornos na literatura, ou "sobre
%% contornos"), coloca na introdução

\section{Revisão de Literatura}
\label{sec:revis-de-liter}

Na década de 1980 Friedmann
\cite{friedmann85:_method_discus_contour,friedmann1987rmc}, Morris
\cite{morris1987cpc}, Marvin e Laprade
\cite{marvin87:_relat_music_contour} e Polansky e Bassein
\cite{polansky92:_possib_impos_melod} iniciaram o desenvolvimento de
teorias de contornos musicais expandindo o trabalho anterior de Seeger
\cite{seeger1960mml}, Kolinski \cite{kolinkski65:_struc_melod_movem} e
Adams \cite{adams1976mct}. Mais tarde essas teorias foram expandidas
por Quinn \cite{quinn97:_fuzzy_exten_theor_contour} e por Beard
\cite{beard2003cmm}.

\section{Metodologia}
\label{sec:metodologia}

%% inserir uma pequena introdução

%% descrever o que será feito

\begin{enumerate}
\item Revisão de literatura sobre contornos melódicos

  Para a realização deste trabalho deverá ser feita uma revisão de
  literatura sobre contornos melódicos e assuntos relacionados.

\item Mapeamento de contornos para elementos musicais

  Trata-se da representação de elementos musicais como altura e
  duração para contornos. Envolve o estudo das particularidades de
  tais elementos como representação numérica de notas, intervalos e
  ritmo, bem como estudo de possíveis operações de contornos.

%% trocar peças didáticas por estudos

\item Composição de peças didáticas para experimento de possibilidades
  com contornos melódicos.

  Duas peças devem ser compostas para experimentar possibilidades de
  uso de contornos melódicos e operações. A primeira, já concluída,
  contém um único contorno melódico e experimenta possibilidades de
  preenchimento entre os pontos do contorno. A segunda, em andamento,
  testa possibilidades de operações e concatenação. Com esses pequenos
  experimentos é possível aprofundar a prática de composição com
  contornos.

\item Desenvolvimento de um programa de computador para processar
  operações com contornos melódicos.

  Um software que lida com contornos melódicos, operações e
  combinações está sendo desenvolvido. Ele retornará representações
  simbólicas, gráficas e musicais de operações de transposição,
  inversão, retrogradação e rotação de um dado contorno. Além disso
  ele permitirá a combinação de operações e sua concatenação. A
  linguagem é Lisp e a plataforma Unix.

%% Dizer pq é importante
  
\item Composição da peça

  Todo o material composicional a ser utilizado na peça deverá ser
  gerado por operações e concatenação de operações de contornos
  melódicos, e por preenchimento desses contornos.

\item Escrita da dissertação
\end{enumerate}

\section{Resultados Esperados}
\label{sec:resultados-esperados}

% incluir a verificação do estado de arte de contornos
Os resultados pretendidos com este trabalho são a obra para quinteto
de madeiras, o software para processamento de contornos melódicos e a
dissertação de mestrado, que incluirá a análise da referida obra.

\section{Estrutura de tópicos da dissertação}
\label{sec:estrutura-de-topicos}

\begin{enumerate}
\item Introdução
\item O estado de arte de contornos melódicos
  \begin{itemize}
  \item Revisão de literatura
  \item Mapeamento de contornos
  \item Ferramentas utilizadas
  \end{itemize}
\item Análise da composição
\item Partitura da composição
\item Conclusão
\item Bibliografia
\end{enumerate}

\section{Cronograma}
\label{sec:cronograma}

\begin{table}[h]
  \begin{tabular}{lllllllll}
    \hline
    Tarefa & Nov./07 & Dez. & Jan./08 & Fev. & Mar. & Abr. & Mai. \\
    \hline
    Revisão de Literatura & X &  &  & & & & & \\
    Experimentos & X & & & & & & & \\
    Software & X & X & X & X & & & &  \\
    Composição da peça & & X & X & X & X & X \\
    Recital & &  & & & & & X \\
    \hline
  \end{tabular}
  \begin{tabular}{llllllll}
    \hline
    Tarefa & Jun./08 & Jul. & Ago. & Set. & Out. & Nov. & Dez. \\
    \hline
    Escrita da Dissertação & X & X & X & X & X & & \\
    Entrega da Dissertação & & & & & & X & \\
    Defesa & & & & & & & X \\
    \hline
  \end{tabular}

  \label{tab:cronograma}
\end{table}


\renewcommand{\refname}{Bibliografia}

\nocite{adams1976mct,albrecht2004ect,alves05:_invar,alves2001cuc,batke2004edm,beard2003cmm,Blackburn2000thesis,buteau2005ama,buteau2003tmm,buteau1999mta,buteau2000csm,cambouropoulos2001mca,cambouropoulos2005pea,clifford1995cse,cook87:_techn_compar_analy,discipio2000ajc,dowling1994mch,dowling1978sac,dowling1971rim,dowling1986mc,edworthy1985mca,forte83:_motiv_rhyth_contour_alto_rhaps,friedmann1987rmc,friedmann85:_method_discus_contour,ghias1995qhm,Hermann1995,hofmannengl1999vpa,ishiyama1996act,kim2000acb,kolinkski65:_struc_melod_movem,lindsay1996ucm,maidin:gam,marvin1988gtm,marvin87:_relat_music_contour,morris93:_new_direc_theor_analy_music_contour,morris1987cpc,parsons1975dta,pauws2002cfo,polansky87mma,polansky92:_possib_impos_melod,prechelt2001imi,quinn2006mcp,quinn97:_fuzzy_exten_theor_contour,raju2002tam,raju2003tqh,ravenscroft2006rcc,music02:_melody,roy2004cbm,schmuckler1999tmm,schubert2006eih,schonberg1967fmc,seeger1960mml,serafine1989idm,sonoda2002bmr,Siu-LanTan04012004,toch1977sfm,toiviainen2002cmm,uitdenbogerd2003pre,weyde2005emr,yi1991tmc,zhu2002smc}

\bibliography{mestrado}
\bibliographystyle{plain}

\end{document}