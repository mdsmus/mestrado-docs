\documentclass{beamer}
\usepackage{beamerthemesplit}
\usepackage{graphicx,url}
\usepackage[brazil]{babel}
\usepackage[utf8]{inputenc}

\mode<presentation>
{
  \usetheme{Ilmenau}
  \setbeamercovered{transparent}
}

\newcommand{\eng}[1]{\textit{#1}}
\newcommand{\obra}{\textit{Em torno da romã}}

\title{\obra{}: aplicações de operações de contorno na composição}
\author{Marcos da Silva Sampaio}
\date{28 de novembro de 2008}

\logo{\includegraphics[scale=.15]{logo-genos}}

\begin{document}

\frame{\titlepage}

\frame{
  \frametitle{Esta apresentação}
  \tableofcontents
}

\section{Introdução}

\frame{
  \frametitle{O que são contornos?}
  \begin{figure}
    \includegraphics[scale=.6]{roma-pura}
    \hspace{1em}
    \includegraphics[scale=.6]{contorno-com-roma}
  \end{figure}
}

\frame{
  \frametitle{Contornos em Música}
  \begin{figure}
    \centering
    \includegraphics{5a-sinfonia}
  \end{figure}

  \begin{figure}
    \centering
    \includegraphics[scale=1.4]{c-3120}
  \end{figure}
}

\frame{
  \frametitle{Semelhança e coerência}
  \begin{figure}
    \centering
    \includegraphics{ly-2031}
    \hspace{1em}
    \includegraphics[scale=1.4]{c-2031}
  \end{figure}
}

\frame{
  \frametitle{Objetivos e justificativa}
  \begin{itemize}
  \item Justificativa
    \begin{itemize}
    \item Coerência
    \item Manipulação por operações
    \item Estudos escassos
    \end{itemize}
  \item Objetivos
    \begin{itemize}
    \item Composição baseada em operações de contornos
    \item Software para processar contornos
    \item Mapeamento de contornos
    \end{itemize}
  \end{itemize}
}

\section{Contornos}

\frame{
  \frametitle{}
}

\section{Goiaba}

\frame{
  \frametitle{}
}

\section{Composição}

\frame{
  \frametitle{}
}

\section{Conclusões}

\frame{
  \frametitle{}
}

\frame[allowframebreaks]{
  \frametitle{Referências}
  \bibliographystyle{alpha}
  \bibliography{melodic-contour,music-perception,composition,music-harmony-and-theory,programs,music-analysis,audio,genos,computer-science}
}

\end{document}
