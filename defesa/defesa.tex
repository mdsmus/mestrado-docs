\documentclass{beamer}
\usepackage{beamerthemesplit}
\usepackage{graphicx,url}
\usepackage[brazil]{babel}
\usepackage[utf8]{inputenc}

\mode<presentation>
{
  \usetheme{Ilmenau}
  \setbeamercovered{transparent}
}

\newcommand{\eng}[1]{\textit{#1}}
\newcommand{\obra}{\textit{Em torno da romã}}

\title{\obra{}: aplicações de operações de contorno na composição}
\author{Marcos da Silva Sampaio}
\date{28 de novembro de 2008}

\logo{\includegraphics[scale=.15]{logo-genos}}

\begin{document}

\frame{\titlepage}

\section{Introdução}

\frame{
  \frametitle{O que são contornos?}
  \begin{enumerate}
  \item Perfis, desenhos ou formatos de objetos.
  \item Podem ser bidimensionais
  \item Podem associar:
    \begin{enumerate}
    \item altura a comprimento
    \item altura a largura
    \item altura a tempo
    \end{enumerate}
  \end{enumerate}
  \includegraphics[scale=.3]{roma-pura}
  \hspace{1em}
  \includegraphics[scale=.3]{contorno-com-roma}
}

\frame{
  \frametitle{Contornos em música}
  \begin{enumerate}
  \item Podem ser associados a altura, densidade, ritmo, homogeneidade
    de timbre, intensidade, etc.
  \item Contornos melódicos: movimentos de altura no tempo.
  \end{enumerate}
  \includegraphics{c-0312}
}

\frame{
  \frametitle{Por que contornos são importantes}
  \begin{enumerate}
  \item Podem ajudar a dar coerência a uma obra musical.
  \item Representam estruturas manipuláveis através de operações como
    inversão e retrogradação.
  \end{enumerate}
}

\frame{
  \frametitle{Justificativa para este trabalho}
  Apesar da possível coerência musical proporcionada por contornos e
  das operações disponíveis, são escassos estudos do uso sistemático
  de operações de contornos e suas combinações na composição musical.
}

\frame{
  \frametitle{Objetivos deste trabalho}
  \begin{enumerate}
  \item Composição de uma obra musical com base em combinações de
    operações de contornos e a produção de seu memorial.
  \item Desenvolvimento de um programa de computador para
    processamento de operações de contornos melódicos.
  \item Mapeamento de contornos para elementos musicais/composicionais.
  \item Levantamento do estado de arte de contornos melódicos.
  \end{enumerate}
}

\frame{
  \frametitle{Metodologia de trabalho}
  \begin{enumerate}
  \item Revisão de literatura.
  \item Mapeamento de contornos para elementos musicais.
  \item Composição de estudos para experimentação de possibilidades
    com contornos.
  \item Desenvolvimento de programa de computador para processamento
    de contornos.
  \item Composição da obra \obra{}.
  \end{enumerate}
}

\section{Contornos}

\frame{
  \frametitle{}
}

\section{Análise da obra}

\frame{
  \frametitle{}
}

\section{Conclusões}

\frame{
  \frametitle{}
}

\end{document}
