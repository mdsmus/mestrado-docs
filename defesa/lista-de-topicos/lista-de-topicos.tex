\documentclass[12pt,a4paper]{article}
\usepackage[utf8]{inputenc}
\usepackage[T1]{fontenc}
\usepackage[brazil]{babel}
\usepackage{hyperref}
\usepackage[top=2.5cm,bottom=2.5cm,left=2.5cm,right=2.5cm]{geometry}
\usepackage{color}

\title{Defesa de mestrado}
\author{Marcos da Silva Sampaio}
\date{28 novembro 2008}

\newcommand{\slide}{\textcolor{red}{(\textit{slide})}}

\begin{document}

\maketitle
\thispagestyle{empty}

\section*{Cumprimentos}

Ao coordenador, aos membros da banca, aos demais presentes

\section{Introdução}

\subsection{Contornos}

\begin{enumerate}
\item O que são contornos? \slide{}
  \begin{itemize}
  \item perfis, desenhos ou formatos de objetos (ex. contorno da romã)
  \item podem ser bidimensionais e associar altura a comprimento,
    largura, etc.
  \end{itemize}
\item Contornos em música. \slide{}

  Associáveis a:
  \begin{itemize}
  \item altura (ex. partitura)
  \item densidade
  \item ritmo
  \item homogeneidade de timbre
  \item intensidade
  \end{itemize}
\item importância do estudo de contornos: melodias com semelhanças
  identificáveis pelo contorno. \slide{}

  (ignora valores absolutos dos elementos)
\item análises a partir de contornos (friedmann, clifford, marvin, beard, eu mesmo)
\item possível coerência em Webern \slide{}
\end{enumerate}

\subsection{A pesquisa}

\begin{enumerate}
\item justificativa \slide{}
  \begin{itemize}
  \item coerência musical
  \item estruturas manipuláveis por operações
  \item estudos escassos de contornos em composição
  \end{itemize}
\item objetivos
  \begin{itemize}
  \item composição baseada em operações de contornos
  \item desenvolvimento de processador de operações de contornos
  \end{itemize}
\end{enumerate}

\section{Contornos}

\begin{enumerate}
\item Teorias desenvolvidas por vários autores
\item Definições \slide{}
  \begin{itemize}
  \item movimento ascendente/descente entre pontos adjacentes 

  \item conjunto ordenado de elementos distintos enumerados de forma
    ascendente

    5a sinfonia / (3 1 2 0)
  \item comparações entre definições \slide{}
    \begin{itemize}
    \item elementos não adjacentes
    \item expansão para outros elementos (dinâmica, densidade, etc)
    \end{itemize}
  \end{itemize}
\item outras considerações
  \begin{itemize}
  \item não trabalho com medida de tempo
  \item definição de operações de contornos
  \end{itemize}
\item representações \slide{}
  \begin{itemize}
  \item de contorno
  \item de operações
  \end{itemize}
\end{enumerate}

\subsection{operações}

\begin{enumerate}
\item implementadas \slide{}

  \texttt{ver no goiaba!!!!}. O que é o goiaba e para que serve?
\item não implementadas
  \begin{itemize}
  \item $INT_n$ \slide{}
  \item redução de contornos de Morris \slide{}
  \end{itemize}
\end{enumerate}

\section{Goiaba}

\begin{enumerate}
\item autoria: marcos e pedro
\item desenvolvimento
  \begin{itemize}
  \item common lisp e sbcl
  \item bottom-up
  \item orientação a objetos
  \end{itemize}
\texttt{ir para o goiaba}
\end{enumerate}

\section{Análise da peça}

\begin{enumerate}
\item características gerais (instrumental e duração) \slide{}
\item foco da composição
  \begin{itemize}
  \item contornos melódicos e não melódicos
  \item proporções
  \item metas composicionais
  \item gestos
  \item motivos
  \end{itemize}
\item materiais utilizados \slide{}
  \begin{itemize}
  \item motivo alfa
  \item contorno P(5 3 4 1 2 0)
  \end{itemize}
\item aspectos formais (sete seções e proporção áurea aproximada)
\item aspectos verticais (escala octatônica)
\item uso de motivos \slide{}
\item uso de contornos \slide{}
  \begin{itemize}
  \item interpolação com expansão (solo oboé seção 5) \slide{}
  \item rotação com expansão (sujeito e cs e seção 6) \slide{}
  \item rotação com retrogradação (solo oboé seção 5) \slide{}
  \item expansão associada à amplitude (segunda seção) \slide{}
  \item redução de contornos (seção 3) 
  \end{itemize}
\item associação a outros parâmetros \slide{}
  \begin{itemize}
  \item andamentos. subconjunto de 5 elementos
  \item densidade. subconjunto de 5 elementos (seção 1)
  \item complexidade das texturas (- + - + -)
  \end{itemize}
\end{enumerate}

\section{Conclusões}

\begin{itemize}
\item discussão (operações que não funcionaram) \slide{}
\item trabalhos futuros \slide{}
  \begin{enumerate}
  \item mapeamento de outros parâmetros (dinâmica x densidade)
  \item teste de outras operações das teorias com pequenos experimentos
  \item uso de contornos em música computacional (outros elementos e parâmetros)
  \item expansão do software goiaba
    \begin{itemize}
    \item versão estável
    \item gui
    \item api fácil de usar
    \item conversão de/para partituras musicais
    \item anteprojeto aceito para doutorado
    \end{itemize}
  \end{enumerate}
\end{itemize}

\end{document}