\documentclass[12pt,a4paper]{article}
\usepackage[utf8]{inputenc}
\usepackage[T1]{fontenc}
\usepackage[brazil]{babel}
\usepackage{hyperref}
\usepackage[top=2.5cm,bottom=2.5cm,left=2.5cm,right=2.5cm]{geometry}

\title{Defesa de mestrado}
\author{Marcos di Silva}
\date{28 novembro 2008}

\begin{document}

\maketitle

\section*{Cumprimentos}

Ao coordenador, aos membros da banca, aos demais presentes

\section{Introdução}

\subsection{Contornos}

\begin{enumerate}
\item O que são contornos?
  \begin{itemize}
  \item perfis, desenhos ou formatos de objetos (ex. contorno da romã)
  \item podem ser bidimensionais e associar altura a comprimento,
    largura, etc.
  \end{itemize}
\item Contornos em música.

  Associáveis a:
  \begin{itemize}
  \item altura (ex. partitura)
  \item densidade
  \item ritmo
  \item homogeneidade de timbre
  \item intensidade
  \end{itemize}
\item melodias com semelhanças identificáveis pelo contorno
  (importância do estudo)
\item análises a partir de contornos (friedmann, clifford, marvin, beard, eu mesmo)
\end{enumerate}

\subsection{A pesquisa}

\begin{enumerate}
\item justificativa
  \begin{itemize}
  \item coerência musical
  \item estruturas manipuláveis por operações
  \item estudos escassos de contornos em composição
  \end{itemize}
\item objetivos
  \begin{itemize}
  \item composição baseada em operações de contornos
  \item desenvolvimento de processador de operações de contornos
  \end{itemize}
\item metodologia
  \begin{itemize}
  \item revisão
  \item mapeamento
  \item composição de experimentos
  \item software
  \item composição
  \end{itemize}
\end{enumerate}

\section{Contornos}

\begin{enumerate}
\item Teorias desenvolvidas por vários autores
\item Definições
  \begin{itemize}
  \item movimento ascendente/descente entre pontos adjacentes 

  \item conjunto ordenado de elementos enumerados de forma ascendente

    5a sinfonia / (3 1 2 0)
  \item comparações entre definições
    \begin{itemize}
    \item elementos não adjacentes
    \item expansão para outros elementos (dinâmica, densidade, etc)
    \end{itemize}
  \end{itemize}
\item outras considerações
  \begin{itemize}
  \item não trabalho com medida de tempo
  \item definição de operações de contornos
  \end{itemize}
\item representações
  \begin{itemize}
  \item de contorno
  \item de operações
  \end{itemize}
\end{enumerate}

\subsection{operações}

\begin{enumerate}
\item implementadas

  \texttt{ver no goiaba!!!!}. O que é o goiaba e para que serve?
\item não implementadas
  \begin{itemize}
  \item $INT_n$
  \item redução de contornos (Adams e Morris)
  \end{itemize}
\end{enumerate}

\section{Goiaba}

\begin{enumerate}
\item autoria: marcos e pedro
\item desenvolvimento
  \begin{itemize}
  \item common lisp e sbcl
  \item bottom-up
  \item orientação a objetos
  \end{itemize}
\texttt{ir para o goiaba}
\end{enumerate}

\section{Análise da peça}

\begin{enumerate}
\item características gerais (instrumental e duração)
\item foco da composição
  \begin{itemize}
  \item contornos melódicos e não melódicos
  \item proporções
  \item metas composicionais
  \item gestos
  \item motivos
  \end{itemize}
\item materiais utilizados
  \begin{itemize}
  \item motivo alfa
  \item contorno P(5 3 4 1 2 0)
  \end{itemize}
\item aspectos formais (sete seções e proporção áurea aproximada)
\item descrição dos gestos das seções (\texttt{no goiaba})
\item aspectos verticais (escala octatônica)
\item uso de motivos
\item uso de contornos
  \begin{itemize}
  \item interpolação com expansão (solo oboé seção 5)
  \item rotação com expansão (sujeito e cs e seção 6)
  \item rotação com retrogradação (solo oboé seção 5)
  \item expansão associada à amplitude (segunda seção)
  \item redução de contornos (seção 3)
  \end{itemize}
\item associação a outros parâmetros
  \begin{itemize}
  \item andamentos. subconjunto de 5 elementos
  \item densidade. subconjunto de 5 elementos (seção 1)
  \item complexidade das texturas (- + - + -)
  \end{itemize}
\end{enumerate}

\section{Conclusões}

\begin{itemize}
\item discussão (operações que não funcionaram)
\item trabalhos futuros
  \begin{enumerate}
  \item mapeamento de outros parâmetros (dinâmica x densidade)
  \item teste de outras operações das teorias com pequenos experimentos
  \item uso de contornos em música computacional (outros elementos e parâmetros)
  \item expansão do software goiaba
    \begin{itemize}
    \item versão estável
    \item gui
    \item api fácil de usar
    \item conversão de/para partituras musicais
    \item anteprojeto aceito para doutorado
    \end{itemize}
  \end{enumerate}
\end{itemize}

\end{document}