\documentclass{article}
\usepackage{ifthen}
\usepackage[utf8,utf8x]{inputenc}
\usepackage[a4paper,top=2cm,bottom=2cm,left=2cm,right=2cm]{geometry}
\usepackage[brazil]{babel}
\usepackage{setspace}
\usepackage{graphicx}
\usepackage{url}
\usepackage{colortbl}

% usar para termos estrangeiros
\newcommand{\eng}[1]{\textit{#1}}

% usar para nomes de obras
\newcommand{\opus}[1]{\textit{#1}}

% usar para nomes de termos
\newcommand{\termo}[1]{\textit{#1}}

\newcommand{\ok}{
  \multicolumn{1}{>{\columncolor[gray]{.6}}c}{}
}

\newcommand{\tri}[1]{
  #1\textsuperscript{o} t
}

\title{Sobre Contornos}
\author{Marcos di Silva}

\begin{document}

\setlength{\parindent}{0cm}
\maketitle
\thispagestyle{empty}

Friedmann \cite{friedmann85:_method_discus_contour} afirma que a
música do século XX tem sido em grande parte analisada em função de
relações de classes de alturas. Ele defende o uso independente de
contornos melódicos para a análise deste repertório. Argumenta que
ouvintes têm uma maior acuidade com contornos que com relações de
classes de altura. Exemplifica a eficiência do uso de contornos com
obras da literatura que têm no contorno o elemento mais
importante. Propõe ferramentas e operações para a autonomia de uma
análise a partir de contornos melódicos.

Friedmann \cite{friedmann85:_method_discus_contour} propõe as seguintes
ferramentas:

\begin{itemize}
\item série de contornos adjacentes \eng{Contour Adjacency Series
    (CAS)}.

  Uma série de sinais '+' e '-' referentes às inclinações positivas e
  negativas de um contorno. Por exemplo, para um contorno ``d c f e'' CAS =
  (-,+,-).
\item classe de contornos \eng{Contour Class (CC)};

  Uma série ordenada de números que representa a relação de alturas
  entre todas as notas de um contorno, na qual a nota mais grave é
  indicada por 0 e a nota mais aguda por n-1, onde n=total de
  notas. Por exemplo, para um contorno ``d c f e'' CC = (1 0 3 2).
\item vetor de classe de contornos adjacentes \eng{Contour Adjacency
    Series Vector (CASV)};

  Um  vetor de dois  dígitos com  a soma  das inclinações  positivas e
  negativas de um contorno. Por exemplo, para  um contorno ``d c f e'' CASV =
  (1,2).
\item intervalo de contorno \eng{Contour Interval (CI)};

  A distância entre dois elementos de uma classe de contorno
  (CC). Por exemplo, em CC (1 0 3 2), a CI de 1-0 é -1, a CI de 0-3 é +3.
\item sucessão de intervalos de contorno \eng{Contour Interval
    Succession (CIS)};

  Uma série de números que representam os intervalos de contorno (CI)
  de uma classe de contorno (CC). Por exemplo, em CC (1 0 3 2), CI = (1 3
  -1).

  %% FIXME: o termo array pode ser traduzido por vetor neste contexto?
\item Vetor de intervalos de contorno \eng{Contour Interval
    Array (CIA)}

  Uma série de números que representa a multiplicidade de intervalos
  de contornos (CI) de uma classe de contorno (CC). Esta série é
  dividida em duas e cada número representa o número de CI's de valor
  1, 2, 3, etc, a depender do tamanho do contorno. A primeira parte da
  série representa CI's positivos e a segunda representa CI's
  negativos. Por exemplo, para CC = (1 0 3 2), CIA = (1 2 1 , 2 0
  0). Esta CC tem um intervalo de valor +1, dois de valor +2, um de
  valor +3, dois de valor -1 e nenhum de valores -2 e -3.
\item vetor de classe de contorno I \eng{Contour Class Vector I
    (CCVI)}
\item vetor de classe de contorno II \eng{Contour Class Vector II
    (CCVII)}
\end{itemize}

\bibliography{mestrado}
\bibliographystyle{plainurl-br}

\end{document}
