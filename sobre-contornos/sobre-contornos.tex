\documentclass{article}
\usepackage{ifthen}
\usepackage[utf8,utf8x]{inputenc}
\usepackage[a4paper,top=2cm,bottom=2cm,left=2cm,right=2cm]{geometry}
\usepackage[brazil]{babel}
\usepackage{setspace}
\usepackage{graphicx}
\usepackage{url}
\usepackage{colortbl}

% usar para termos estrangeiros
\newcommand{\eng}[1]{\textit{#1}}

% usar para nomes de obras
\newcommand{\opus}[1]{\textit{#1}}

% usar para nomes de termos
\newcommand{\termo}[1]{\textit{#1}}

\newcommand{\ok}{
  \multicolumn{1}{>{\columncolor[gray]{.6}}c}{}
}

\newcommand{\tri}[1]{
  #1\textsuperscript{o} t
}

\title{Sobre Contornos}
\author{Marcos di Silva}

\begin{document}

\setlength{\parindent}{0cm}
\maketitle

Friedmann \cite{friedmann85:_method_discus_contour} afirma que a
música do século XX tem sido em grande parte analisada em função de
relações de classes de alturas. Ele defende o uso independente de
contornos melódicos para a análise deste repertório. Argumenta que
ouvintes têm uma maior acuidade com contornos que com relações de
classes de altura. Exemplifica a eficiência do uso de contornos com
obras da literatura que têm no contorno o elemento mais
importante. Propõe ferramentas e operações para a autonomia de uma
análise a partir de contornos melódicos.

Friedmann \cite{friedmann85:_method_discus_contour} propõe as seguintes
ferramentas:

\begin{itemize}
\item série de contornos adjacentes \eng{Contour Adjacency Series
    (CAS)};
\item classe de contornos \eng{Contour Class (CC)};
\item vetor de classe de contornos adjacentes \eng{Contour Adjacency
    Series Vector (CASV)};
\item intervalo de contorno \eng{Contour Interval (CI)};
\item sucessão de intervalos de contorno \eng{Contour Interval
    Succession (CIS)};
\item \eng{array} de intervalos de contorno \eng{Contour Interval
    Array (CIA)}
\item vetor de classe de contorno I \eng{Contour Class Vector I
    (CCVI)}
\item vetor de classe de contorno II \eng{Contour Class Vector II
    (CCVII)}
\end{itemize}

\bibliography{mestrado}
\bibliographystyle{plainurl-br}

\end{document}