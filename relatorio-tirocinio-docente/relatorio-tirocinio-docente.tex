\documentclass{article}
\usepackage{ifthen}
\usepackage[utf8,utf8x]{inputenc}
\usepackage[a4paper,top=2cm,bottom=2cm,left=2cm,right=2cm]{geometry}
\usepackage[brazil]{babel}
\usepackage{setspace}
\usepackage{graphicx}
\usepackage{url}
\usepackage{colortbl}

% usar para termos estrangeiros
\newcommand{\eng}[1]{\textit{#1}}

% usar para nomes de obras
\newcommand{\opus}[1]{\textit{#1}}

% usar para nomes de termos
\newcommand{\termo}[1]{\textit{#1}}

\newcommand{\ok}{
  \multicolumn{1}{>{\columncolor[gray]{.6}}c}{}
}

\newcommand{\tri}[1]{
  #1\textsuperscript{o} t
}

\begin{document}

\setlength{\parindent}{0cm}

%% cabecalho
\large
UNIVERSIDADE FEDERAL DA BAHIA \\
ESCOLA DE MÚSICA \\
PROGRAMA DE PÓS-GRADUAÇÃO \\
MESTRADO EM COMPOSIÇÃO \\
ORIENTADOR: Pedro Kröger \\
ALUNO: Marcos da Silva Sampaio \\
DATA: \today

\thispagestyle{empty}
\vspace{1cm}
\begin{center}{
    \Huge \textbf{Relatório de Tirocício Docente} \\
}
\vspace{12pt}
{\Large Disciplina: Composição III (MUS110)}

\end{center}
\vspace{1cm}

\section{Introdução e Objetivos}
\label{sec:introducao}

%% parágrafos soltos:

A disciplina de Composição III tem periodicidade anual. Este
relatório, porém, abrange apenas o segundo semestre de trabalho.

Esta disciplina ainda não tem conteúdo programático aprovado pelo
departamento CLEM e, de acordo com sua ementa, ela deve ``fornecer aos
alunos conhecimento nos elementos básicos da composição e
instrumentação''. Os conteúdos então devem ser escolhidos pelo
professor.

O programa seguido foi sugerido pelo orientador da atividade,
prof. Pedro Kröger.


O tema trabalhado no segundo semestre foi ``Exposição de uma idéia
musical na composição''.

Os conteúdos envolvidos foram a prática de exposição de idéias
musicais curtas (até 20 compassos) e análise da exposição de idéias
nos quartetos opus 18 e 59 de Beethoven.

\section{Metodologia}
\label{sec:metodologia}

Análise e audição de quartetos de cordas de vários compositores.

Claude Debussy, Maurice Ravel, Béla Bartók , Heitor Villa-Lobos,
Wolfgang A. Mozart, Krzysztof Penderecki, Derek Bermel, John Cage,
Jacques Morelembaum (arranjo popular),

Composição de um gesto musical a partir de uma única idéia
composicional.

Análise dos compassos iniciais de um movimento de um quarteto de
Beethoven e apresentação em sala de aula.

Trabalho de final de semestre realizado em um mês com orientação
semanal com o professor e discussão com a turma.

Trabalho final realizado em sala de aula.

Audição extra: Penderecki.

\section{Recursos}
\label{sec:recursos}

\section{Resultados Alcançados}
\label{sec:result-alcanc}

\section{Avaliação}
\label{sec:avaliacao}

A avaliação dos alunos pelo professor foi feita a partir dos trabalhos
pedidos e da participação nas discussões do trabalho final. Exceto
pela composição de final de semestre, a simples entrega das atividades
garantiu a nota máxima para os alunos. No trabalho final a duração
pedida foi estritamente observada e os trabalhos fora das
especificações foram penalizados.

De acordo com a avaliação que os alunos fizeram do curso, os pontos
positivos foram a análise dos quartetos de Beethoven, a orientação
semanal e discussão dos trabalhos individuais em sala, e o trabalho
final realizado inteiramente em sala de aula. O único ponto negativo
apontado foi o fato dos trabalhos em sala de aula terem acontecido
apenas uma vez em todo o ano.

\section{Conclusão}
\label{sec:conclusao}



\renewcommand{\refname}{Bibliografia}

\nocite{Beethoven1970,Heussenstamm1987,adler89:_study_orches,Kennan1997,stone80:_music_notat_twent_centur}

\bibliography{mestrado}
\bibliographystyle{plainurl-br}

\end{document}