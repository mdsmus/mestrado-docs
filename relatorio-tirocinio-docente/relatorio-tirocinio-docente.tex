\documentclass{article}
\usepackage{ifthen}
\usepackage[utf8,utf8x]{inputenc}
\usepackage[a4paper,top=2cm,bottom=2cm,left=2cm,right=2cm]{geometry}
\usepackage[brazil]{babel}
\usepackage{setspace}
\usepackage{graphicx}
\usepackage{url}
\usepackage{colortbl}

% usar para termos estrangeiros
\newcommand{\eng}[1]{\textit{#1}}

% usar para nomes de obras
\newcommand{\opus}[1]{\textit{#1}}

% usar para nomes de termos
\newcommand{\termo}[1]{\textit{#1}}

\newcommand{\ok}{
  \multicolumn{1}{>{\columncolor[gray]{.6}}c}{}
}

\newcommand{\tri}[1]{
  #1\textsuperscript{o} t
}

\begin{document}

\setlength{\parindent}{0cm}

%% cabecalho
\large
UNIVERSIDADE FEDERAL DA BAHIA \\
ESCOLA DE MÚSICA \\
PROGRAMA DE PÓS-GRADUAÇÃO \\
MESTRADO EM COMPOSIÇÃO \\
ORIENTADOR: Pedro Kröger \\
ALUNO: Marcos da Silva Sampaio \\
DATA: \today

\thispagestyle{empty}
\vspace{1cm}
\begin{center}{
    \Huge \textbf{Relatório de Tirocício Docente} \\
}
\vspace{12pt}
{\Large Disciplina: Composição III (MUS110)}

\end{center}
\vspace{1cm}

\section{Introdução e Objetivos}
\label{sec:introducao}

%% parágrafos soltos:

A disciplina de Composição III tem periodicidade anual. Este
relatório, porém, abrange apenas o trabalho realizado no segundo
semestre.

Esta disciplina ainda não tem conteúdo programático aprovado pelo
departamento CLEM e, de acordo com sua ementa, ela deve ``fornecer aos
alunos conhecimento nos elementos básicos da composição e
instrumentação''. Os conteúdos então devem ser escolhidos pelo
professor. O programa seguido foi sugerido pelo orientador da
atividade, prof. Pedro Kröger.

O tema trabalhado no segundo semestre foi ``Exposição de uma idéia
musical na composição''.

%% explicar o que idéia musical significa aqui.

Os conteúdos envolvidos foram a prática de exposição de idéias
musicais curtas (até 20 compassos) e análise da exposição de idéias
nos quartetos opus 18 e 59 de Beethoven.

\section{Metodologia}
\label{sec:metodologia}

%% fazer parágrafo de introdução

Composição prévia de um pequeno gesto musical de até vinte compassos a
partir de uma única idéia composicional. Este trabalho foi feito como
uma sondagem antes da apresentação do conteúdo. O prazo de entrega foi
de duas semanas e houve um dia de orientação para esclarecimento de
dúvidas.

Apreciação e análise do primeiro movimento do Quarteto de Cordas nº 1
op.18 de Beethoven. Esta análise se restringiu à observação da
exposição das idéias musicais. Repetição, variação, imitação,
articulação, dinâmica, orquestração, textura, uso de motivos e
aproveitamento do material composicional foram analisados em
diferentes trechos da obra citada.

Apresentação individual dos alunos em sala de análise dos primeiros
compassos de todos os quartetos de cordas do opus 18 de Beethoven
(quartetos 1 ao 6). Assim como o trabalho de sondagem, o prazo de
preparação do trabalho foi de duas semanas e houve também um dia de
orientação para o esclarecimento de dúvidas.

Apreciação e análise dos primeiros movimentos de quartetos de cordas
de outros compositores com a mesma metodologia empregada no Op. 18 de
Beethoven. Os compositores foram, além de Beethoven (op.59), Debussy,
Ravel, Villa-Lobos, Mozart (KV 465), Penderecki, Derek Bermel, John
Cage e Jacques Morelembaum\footnote{Música de Maurício Maestro e
  arranjo Jacques Morelembaum para quarteto de cordas e voz.}.

%% explicar melhor o trabalho principal
Composição principal do semestre, desenvolvida em um mês com
orientação semanal com o professor e discussão com a turma. Os
requisitos para esse trabalho foram a formação de quarteto de cordas e
a duração entre três e cinco minutos.

Composição final realizada em sala de aula. Requisitos:
\begin{itemize}
\item Formação de quarteto de cordas.
\item Duração de dez a vinte seguntos.
\item Limitação de idéias: foi pedida uma idéia musical bastante curta
  composta, de preferência antes do ingresso dos alunos no curso de
  Composição.
\item Gestual: dois pequenos gestos interligados.
\item Orquestração: foi pedido que a composição se iniciasse e
  finalizasse com um instrumento solo, e que em algum ponto ao longo
  da composição houvesse um tutti
\item Registros: foi pedido que ao longo da composição todo o registro
  do quarteto fosse utilizado.
\end{itemize}

Audição extra: Penderecki (Paixão segundo Lucas).

\section{Recursos}
\label{sec:recursos}

\section{Resultados Alcançados}
\label{sec:result-alcanc}

\section{Avaliação}
\label{sec:avaliacao}

A avaliação dos alunos pelo professor foi feita a partir dos trabalhos
pedidos e da participação nas discussões do trabalho final. Exceto
pela composição de final de semestre, a simples entrega das atividades
garantiu a nota máxima para os alunos. No trabalho final a duração
pedida foi estritamente observada e os trabalhos fora das
especificações foram penalizados.

De acordo com a avaliação que os alunos fizeram do curso, os pontos
positivos foram a análise dos quartetos de Beethoven, a orientação
semanal e discussão dos trabalhos individuais em sala, e o trabalho
final realizado inteiramente em sala de aula. O único ponto negativo
apontado foi o fato dos trabalhos em sala de aula terem acontecido
apenas uma vez em todo o ano.

\section{Conclusão}
\label{sec:conclusao}


\renewcommand{\refname}{Bibliografia}

\nocite{Beethoven1970,Heussenstamm1987,adler89:_study_orches,Kennan1997,stone80:_music_notat_twent_centur}

\bibliography{mestrado}
\bibliographystyle{plainurl-br}

\end{document}