\documentclass[12pt]{article}
\usepackage{ifthen}
\usepackage[utf8,utf8x]{inputenc}
\usepackage[a4paper,top=2cm,bottom=2cm,left=2cm,right=2cm]{geometry}
\usepackage[brazil]{babel}
\usepackage{setspace}
\usepackage{graphicx}
\usepackage{url}
\usepackage{colortbl}

% usar para termos estrangeiros
\newcommand{\eng}[1]{\textit{#1}}

% usar para nomes de obras
\newcommand{\opus}[1]{\textit{#1}}

% usar para nomes de termos
\newcommand{\termo}[1]{\textit{#1}}

\newcommand{\ok}{
  \multicolumn{1}{>{\columncolor[gray]{.6}}c}{}
}

\newcommand{\tri}[1]{
  #1\textsuperscript{o} t
}

\begin{document}

\setlength{\parindent}{0cm}
\setlength\parskip{1ex}

%% cabecalho
\large
UNIVERSIDADE FEDERAL DA BAHIA \\
ESCOLA DE MÚSICA \\
PROGRAMA DE PÓS-GRADUAÇÃO \\
MESTRADO EM COMPOSIÇÃO \\
ORIENTADOR: Pedro Kröger \\
ALUNO: Marcos da Silva Sampaio \\
DATA: \today

\thispagestyle{empty}
\vspace{1cm}
\begin{center}{
    \Huge \textbf{Relatório de Tirocício Docente} \\
}
\vspace{12pt}
{\Large Disciplina: Composição III (MUS110)}

\end{center}
\vspace{1cm}

\section{Introdução e Objetivos}
\label{sec:introducao}

A disciplina de Composição III tem periodicidade anual. Este
relatório, porém, abrange apenas o trabalho realizado no segundo
semestre.

Esta disciplina tem um conteúdo programático antigo e que não
especifica muita coisa. De acordo com sua ementa, ela deve ``fornecer
aos alunos conhecimento nos elementos básicos da composição e
instrumentação''. Os conteúdos então devem ser escolhidos pelo
professor. O programa seguido foi sugerido pelo orientador da
atividade, prof. Pedro Kröger.

O tema trabalhado no segundo semestre foi a ``Exposição de uma idéia
musical na composição''.

No contexto deste curso pode-se entender por idéia musical todo
elemento que colabore com a estruturação de uma peça como um
determinante composicional. Uma idéia pode ser então um simples
motivo, um tema, uma sucessão de acordes, uma textura, ou qualquer
outro elemento que possa ser usado como determinante composicional.
% Exemplos de outros elementos como determinantes composicionais podem
% ser vistos em Babbitt \cite{babbitt1960tti,babbitt1961ssc}.

Os conteúdos envolvidos foram a prática de exposição de idéias
musicais curtas, expostas em até 20 compassos, e a análise da
exposição de idéias nos quartetos números 1 a 5 do opus 18 e o
primeiro 59 de Beethoven.

\section{Metodologia}
\label{sec:metodologia}

A metodologia empregada durante o curso foi escolhida por favorecer um
aprendizado prático dos conteúdos. Ela consistiu no seguinte:

\begin{enumerate}
\item Composição prévia de um pequeno gesto musical de até vinte
  compassos a partir de uma única idéia composicional. Este trabalho
  foi feito como uma sondagem antes da apresentação do conteúdo. O
  prazo de entrega foi de duas semanas e houve um dia de orientação
  para esclarecimento de dúvidas.
\item Apreciação e análise do primeiro movimento do Quarteto de Cordas
  nº 1 op.18 de Beethoven. Esta análise se restringiu à observação da
  exposição das idéias musicais. Repetição, variação, imitação,
  articulação, dinâmica, orquestração, textura, uso de motivos e
  aproveitamento do material composicional foram analisados em
  diferentes trechos da obra citada.
\item Apresentação individual dos alunos em sala de análise dos
  primeiros compassos de todos os quartetos de cordas do opus 18 de
  Beethoven (quartetos 1 ao 5). Assim como o trabalho de sondagem, o
  prazo de preparação do trabalho foi de duas semanas e houve também
  um dia de orientação para o esclarecimento de dúvidas.
\item Apreciação e análise dos primeiros movimentos de quartetos de
  cordas de outros compositores com a mesma metodologia empregada no
  Op. 18 de Beethoven. Os compositores foram, além de Beethoven
  (Quarteto n°.7 op.59), Debussy, Ravel, Villa-Lobos, Mozart (KV 465),
  Bartók, Penderecki, Derek Bermel, John Cage e Maurício Maestro.

\item Composição principal do semestre, desenvolvida em um mês com
  orientação semanal do professor e discussão com a turma. Os
  requisitos para esse trabalho foram a formação de quarteto de cordas
  e a duração entre três e cinco minutos.

\item Composição final realizada em sala de aula. Requisitos:
  \begin{itemize}
  \item Formação de quarteto de cordas.
  \item Duração de 10 a 20 segundos.
  \item Limitação de idéias: foi pedida uma idéia musical bastante curta
    composta, de preferência antes do ingresso dos alunos no curso de
    Composição.
  \item Gestual: dois pequenos gestos interligados.
  \item Orquestração: foi pedido que a composição se iniciasse e
    finalizasse com um instrumento solo, e que em algum ponto ao longo
    da composição houvesse um tutti
  \item Registros: foi pedido que ao longo da composição todo o registro
    do quarteto fosse utilizado.
  \end{itemize}

\item Audição extra: Penderecki (Paixão de Cristo segundo São Lucas).
\end{enumerate}

\section{Recursos}
\label{sec:recursos}

Os recursos utilizados foram equipamento de som, quadro branco, piano
e violino. Dois alunos tocaram o violino.

\section{Resultados Alcançados}
\label{sec:result-alcanc}

Os principais resultados alcançados pelos alunos foram:

\begin{enumerate}
\item Aprendizado prático da exposição de uma idéia musical.
\item Aprendizado de características de instrumentos de cordas
  importantes para a composição.
\item Aprendizado de elementos básicos de composição como uso de
  variação, imitação, contraste, repetição, articulação, dinâmica,
  orquestração, textura, uso de motivos e aproveitamento do material
  composicional no desenvolvimento de uma idéia composicional.
\item Aprendizado em nível inicial de como analisar motivos, forma,
  orquestração e harmonia de uma peça. Nesta fase os alunos ainda não
  têm costume de analisar música. Eles viram e praticaram, ainda que
  de forma superficial, que podem fazer análise de harmonia,
  orquestração, forma e motivos.
\item Melhor leitura de partitura. Muitos alunos só tinham prática de
  leitura de partituras para instrumento solo.
\item Aprendizado de termos musicais. Os alunos enriqueceram seu
  vocabulário técnico, ainda em fase inicial. Aprenderam termos
  referentes a articulação (legato, stacatto), instrumentação (talão,
  sul tasto, sul ponticello), etc.
\item Conhecimento de obras importantes da literatura. A maioria dos
  alunos não conhecia as peças tocadas, exceto a de Villa-Lobos,
  sugerida por um deles.
\end{enumerate}

\section{Avaliação}
\label{sec:avaliacao}

A avaliação dos alunos pelo professor foi feita a partir dos trabalhos
pedidos e da participação nas discussões do trabalho final. Exceto
pela composição de final de semestre, a simples entrega das atividades
garantiu a nota máxima para os alunos. No trabalho final a duração
pedida foi estritamente observada e os trabalhos fora das
especificações foram penalizados.

Todos os alunos que freqüentaram as aulas foram aprovados, embora
alguns com média mais baixa que outros. Todos têm condições de cursar
a disciplina Composição IV.

De acordo com a avaliação que os alunos fizeram do curso, os pontos
positivos foram a análise dos quartetos de Beethoven, a orientação
semanal e discussão dos trabalhos individuais em sala, e o trabalho
final realizado inteiramente em sala de aula. O único ponto negativo
apontado pelos alunos foi o fato dos trabalhos em sala de aula terem
acontecido apenas uma vez em todo o ano.

\section{Conclusão}
\label{sec:conclusao}

O trabalho realizado nesta disciplina enfocou basicamente dois pontos:
a exposição de idéias composicionais, e a escrita para quarteto de
cordas. Com este trabalho os alunos tiveram um bom amadurecimento em
suas composições. Passaram a ter preocupações com forma, coerência no
uso dos materiais e com o aproveitamento dos recursos dos
instrumentos.

A metodologia empregada na segunda parte do semestre --- aulas
baseadas na orientação dos trabalhos dos alunos --- leva à expansão
dos conteúdos de aula, uma vez que os problemas encontrados na
composição de uma peça extrapolam o planejamento inicial dos conteúdos
das aulas. Isso torna tal metodologia interessante para o ensino desta
disciplina, apesar de exigir bastante experiência do professor.

Para o professor os pontos mais importantes do seu próprio aprendizado
foram o entendimento dos objetivos dos alunos em suas composições, a
valorização de cada fragmento levado pelos alunos, e a orientação sem
interferência na idéia composicional dos alunos.

\renewcommand{\refname}{Bibliografia}

\nocite{Beethoven1970,Heussenstamm1987,adler89:_study_orches,Kennan1997,stone80:_music_notat_twent_centur,casella50:_la}

\bibliography{mestrado}
\bibliographystyle{plainurl-br}

\section{Anexos}
\label{sec:anexos}

\subsection{Plano de aula de análise}
\label{sec:plano-de-aula-analise}

Análise do Quarteto n\ro. 1 de Beethoven (Op.18), mov.1 ---
  Parte 1 --- 14/08/2007 --- 2007.2

\subsubsection{Conteúdo}
Análise do desenvolvimento das idéias composicionais de L.V.Beethoven
no Quarteto n\ro. 1 (Op.18), primeiro movimento.

\subsubsection{Objetivo geral}

Levar o aluno a compreender como uma idéia composicional pode ser
desenvolvida e a aproveitar este conhecimento para sua composição.

\subsubsection{Objetivos específicos}

\begin{enumerate}
\item Levar o aluno a expandir seu repertório de possibilidades de
  desenvolvimento de idéias composicionais para aplicação em suas
  próprias obras.
\item Levar o aluno a compreender como uma peça pode ser composta com
  um número mínimo de idéias composicionais.
\item Levar o aluno a compreender diversas maneiras de apresentação de
  uma idéia composicional.
\item Levar o aluno a compreender diversas formas de desenvolvimento
  de uma idéia composicional.
\item Levar o aluno a avaliar criticamente o desenvolvimento de uma
  idéia composicional.
\item Levar o aluno a abstrair outras possibilidades de
  desenvolvimento de uma idéia composicional e especular a respeito do
  que o compositor poderia ter fazer.
\item Levar o aluno a praticar leitura de partitura com claves de sol,
  fá e dó na terceira linha.
\item Levar o aluno a conhecer escrita idiomática para quarteto de
  cordas.
\item Levar o aluno a compreender possibilidades para uso dos
  parâmetros estudados no primeiro semestre (dinâmica, intensidade,
  timbre, articulação, textura, gestual, duração, ritmo e aspecto
  vertical).
\item Expandir o conhecimento de repertório do aluno.
\end{enumerate}

\subsubsection{Metodologia}
Dinâmica de audição e leitura de partitura associada a perguntas
dirigidas ao conteúdo que se deve alcançar e especulações sobre passos
dados pelo compositor.

\begin{enumerate}
\item Apreciação inicial do movimento a ser analisado através de
  audição e leitura da partitura (leitura opcional).
\item Identificação da primeira idéia apresentada pelo compositor.
\item Verificação de semelhanças entre os primeiros microgestos
  apresentados pelo compositor.
\item Introdução da idéia de repetição, variação e contraste.
\item Listagem de possíveis parâmetros para gerar variação em uma
  idéia composicional.
\item Avaliação de semelhanças e diferenças entre cada microgesto
  apresentado pelo compositor.
\item Identificação de outras idéias composicionais.
\item Especulação do tipo ``O que o compositor poderia fazer agora?''.
\item Esboço de um esquema analítico para as idéias estudadas.
\end{enumerate}

\subsubsection{Recursos}
Voz, piano, aparelho de som, partitura e gravação da obra Op.18 de
Beethoven, e quadro branco.

\subsubsection{Avaliação}
Observação do gestual e da participação dos alunos.

\subsubsection{Duração}
1 hora e 40 minutos

\subsection{Trabalhos de alunos}
\label{sec:trabalhos-de-alunos}

\begin{enumerate}
\item Trabalho de análise de Emerson Mattos
\item Trabalho de composição de Vítor Rios
\end{enumerate}
\end{document}