\documentclass{article}
\usepackage{ifthen}
\usepackage[utf8,utf8x]{inputenc}
\usepackage[a4paper,top=2cm,bottom=2cm,left=2cm,right=2cm]{geometry}
\usepackage[brazil]{babel}
\usepackage{setspace}
\usepackage{graphicx}
\usepackage{url}
\usepackage{colortbl}

% usar para termos estrangeiros
\newcommand{\eng}[1]{\textit{#1}}

% usar para nomes de obras
\newcommand{\opus}[1]{\textit{#1}}

% usar para nomes de termos
\newcommand{\termo}[1]{\textit{#1}}

\newcommand{\ok}{
  \multicolumn{1}{>{\columncolor[gray]{.6}}c}{}
}

\newcommand{\tri}[1]{
  #1\textsuperscript{o} t
}

\begin{document}

\setlength{\parindent}{0cm}

%% cabecalho
\large
UNIVERSIDADE FEDERAL DA BAHIA \\
ESCOLA DE MÚSICA \\
PROGRAMA DE PÓS-GRADUAÇÃO \\
MESTRADO EM COMPOSIÇÃO \\
ORIENTADOR: Pedro Kröger \\
ALUNO: Marcos da Silva Sampaio \\
DATA: \today

\thispagestyle{empty}
\vspace{1cm}
\begin{center}{
    \Huge \textbf{Relatório de Tirocício Docente} \\
}
\vspace{12pt}
{\Large Disciplina: Composição III (MUS110)}

\end{center}
\vspace{1cm}

\section{Introdução}
\label{sec:introducao}

\section{Conteúdos}
\label{sec:conteudos}

\section{Objetivos}
\label{sec:objetivos}

O principal objetivo do trabalho foi levar os alunos a compreenderem e
praticarem a exposição de uma idéia musical.

\section{Justificativa}
\label{sec:justificativa}

\section{Metodologia}
\label{sec:metodologia}

\section{Recursos}
\label{sec:recursos}

\section{Avaliação}
\label{sec:avaliacao}

\section{Cronograma}
\label{sec:cronograma}

\renewcommand{\refname}{Bibliografia}

\nocite{Beethoven1970,Heussenstamm1987,adler89:_study_orches,Kennan1997,stone80:_music_notat_twent_centur}

\bibliography{mestrado}
\bibliographystyle{plainurl-br}

\end{document}