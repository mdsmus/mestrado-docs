\documentclass[12pt]{article}
\usepackage{doc-ppgmus}

\begin{document}
\cabecalho{524}{Pedro Kröger} 

\titulo{Relatório de atividades do semestre}

%% trabalho com semelhanças: comparação de fragmentos de obras
%% mudança de tema: semelhanças para contornos
%% trabalho com preenchimento de contornos

Os objetivos da disciplina de Seminários em Composição I foram o
estudo de questões composicionais relacionadas ao anteprojeto de
pesquisa apresentado na seleção para o mestrado e a composição de uma
peça, preferencialmente, das propostas no anteprojeto. Este trabalho
tutorial foi possível graças ao tamanho reduzido da turma, composta
por apenas dois alunos.

O anteprojeto que submeti previa a composição de cinco peças curtas
envolvendo semelhanças de estruturas como contornos melódicos,
texturas, ritmo, etc. Ao longo do curso o rumo da pesquisa foi
alterado. Por isso foram feitas duas atividades diferentes ao longo do
semestre, uma referente a cada tema pensado para o mestrado.

A primeira atividade foi o estudo de semelhanças de estruturas em
obras da literatura, e a segunda o estudo das possibilidades de
preenchimento de segmentos de um dado contorno melódico.

A primeira atividade consistiu na busca na literatura por exemplos de
fragmentos semelhantes entre si, sua catalogação e identificação dos
parâmetros comuns e diferentes. Para essa tarefa foi feita uma lista
de obras e outra de parâmetros.

As obras estudadas foram:
\begin{itemize}
\item Bartók. Concerto para orquestra,
\item Stravinsky. A Sagração da Primavera,
\item Rimsky-Korsakov. Scheherazade,
\item Ravel. Daphnis \& Chloe,
\item Debussy. Prelude Ce qu'a vu le vent d'ouest,
\item Beethoven. 5$^{a}$ Sinfonia e
\item Mozart. Sonata para piano KV331.
\end{itemize}

Os parâmetros previamente escolhidos foram altura, duração,
intensidade, ritmo, métrica, contorno, textura, timbre, articulação e
orquestração.

Ouvimos e lemos as peças selecionadas e então escolhemos algumas
estruturas como contornos melódicos ascendentes e
ascendentes-descendentes, baixo d'alberti e articulação
legato-staccato para comparação. As comparações foram feitas tanto
entre estruturas de uma mesma obra quanto estruturas de obras
diferentes.

Verificamos também combinações de estruturas, como por exemplo
contorno ascendente combinado com dinâmica ascendente. Após uma
triagem preliminar decidimos estudar apenas contornos melódicos
associados a articulação e dinâmica, deixando de lado textura,
métrica, ritmo, timbre e orquestração.

Após as comparações iniciais foi observado que dois fragmentos que
contêm contornos melódicos semelhantes (ascendente-descendente, por
exemplo) guardam um número muito maior de diferenças entre si do que
de semelhanças. No \opus{Concerto para Orquestra} de Bartók os
fragmentos dos compassos 489--515 e 558--565 do quinto movimento foram
comparados. Como semelhança podemos citar a repetição do contorno
ascendente-descendente, a articulação legato em ambos, o padrão
melódico escalar do contorno (tom, semitom) e o ritmo de quiáltera de
colcheias. Como diferenças citamos o âmbito, a dinâmica, a
instrumentação, a distância entre cada movimento
ascendente-descendente (nos primeiros compassos os movimento são
justapostos, nos últimos há pausa entre eles), material em
\eng{foreground} e \eng{background}, andamento, métrica e número de
notas nos movimentos ascendentes-descendentes.

A composição de uma das peças propostas no anteprojeto foi iniciada e
abandonada após a mudança de rumo na pesquisa. O estudo das
semelhanças e diferenças feito no primeiro período colaboraram com
essa mudança de rumo. A nova proposta consiste na composição de uma
peça a partir de combinações de operações de inversão, rotação,
retrogradação, expansão e preenchimento de contornos melódicos.

A segunda atividade da disciplina foi o mapeamento de contornos e uma
composição para testes de preenchimentos de contornos. Este mapeamento
consistiu no entendimento matemático de um contorno. As
características musicais foram ignoradas para que matematicamente
pudéssemos compreender todas as possibilidades de definição e de
operações existentes com contornos. Este estudo matemático se
restringiu ao estudo de coordenadas cartesianas e funções do primeiro
grau. Só após a conclusão desta etapa transpomos o conhecimento
matemático para a música e observamos positivamente a viabilidade do
estudo. Foi proposta ainda uma representação simbólica de contornos
melódicos baseada em coordenadas cartesianas. Os testes feitos com
preenchimento de contornos e a representação simbólica são explicados
em relatório à parte.

\end{document}
