\documentclass[12pt]{article}
\usepackage{doc-ppgmus}

\begin{document}
\cabecalho{524}{Pedro Kr�ger} 

\titulo{Relat�rio: composi��o de \opus{Contornando}}

%%% id�ias:
%% contorno como evento que pode ser aninhado
%% contorno resultante
%% "subcontorno"
%% preenchimento pode variar em fun��o de par�metros:
%% ritmo 
%% articula��o
%% altura

\section{Contorno utilizado}
\label{sec:contorno-utilizado}

O contorno utilizado para esta pe�a (c.f. figura \ref{fig:contorno-1})
pode ser representado por:

\begin{figure}[h]
  \centering
\begin{verbatim}
(event 'contorno-1
 :altura 4
 :duracao 4
 :tempo 60
 ((0 0)(2 3)(3 2)))
\end{verbatim}
  \caption{Representa��o do contorno}
  \label{fig:repres-cont}
\end{figure}

\begin{figure}[h]
  \centering
  \includegraphics[scale=1]{contorno-1}
  \caption{Contorno 1}
  \label{fig:contorno-1}
\end{figure}

\section{Tipos de preenchimento}
\label{sec:tipos-de-preench}

O preenchimento dos segmentos de um dado contorno pode ser feito de
v�rias maneiras. A forma mais simples � usar uma nota em cada
extremidade dos segmentos.

\bibliographystyle{mdschicago}
\bibliography{mestrado}

\end{document}