\documentclass[12pt]{article}
\usepackage{doc-ppgmus}

\begin{document}
\graphicspath{{figs-out/}}
\cabecalho{524}{Pedro Kröger} 

\titulo{Relatório: composição de \opus{Como é que se preenche um
    segmento melódico?}}

%%% idéias:
%% contorno como evento que pode ser aninhado
%% contorno resultante
%% "subcontorno"
%% preenchimento pode variar em função de parâmetros:
%% ritmo 
%% articulação
%% altura

\section{Introdução}
\label{sec:introducao}

A peça \opus{Como é que se preenche um segmento melódico?} é um
exercício cuja realização teve como objetivo experimentar
possibilidades no preenchimento de contornos melódicos. Um dado
contorno representado graficamente pela figura \ref{fig:contorno-1}
tem dois segmentos e três pontos extremos. O preenchimento deste
contorno com notas pode ser feito de várias maneiras. O menor número
possível de notas é três, com uma em cada extremo. Pode-se colocar
várias outras notas em cada segmento. Essas notas podem ser longas e
ligadas, podem ser curtas alternadas com pausas ou um misto das duas
coisas. Elas podem ter ou não a mesma diferença de altura entre si.
Para a composição da peça foi ainda utilizada uma representação
simbólica de contornos melódicos em desenvolvimento pelo compositor.

\begin{figure}[h]
  \centering
  \includegraphics[scale=1]{contorno-1}
  \caption{Representação gráfica dos contornos utilizados}
  \label{fig:contorno-1}
\end{figure}

\section{Representação simbólica de contornos melódicos}
\label{sec:repr-simb-de}

A representação simbólica de contornos melódicos utilizada nesta peça
consiste na representação das extremidades dos segmentos dos contornos
com coordenadas cartesianas em que \texttt{y} representa altura e
\texttt{x} tempo (veja figuras \ref{fig:repres-cont1},
\ref{fig:repres-cont2} e \ref{fig:repres-cont3}). Os valores dessas
coordenadas passam a ter sentido musical a partir da inserção dos
parâmetros altura, duração e tempo. A altura, indicada em semitons é
multiplicada por \texttt{y}, a duração, indicada como pulso mínimo (1
para semibreve, 2 para mínima, 4 para semínima, etc), é multiplicada
por \texttt{x}, e o tempo apenas indica o andamento. Este último
parâmetro é útil na comparação entre contornos. Esta representação
está em fase de testes e até o momento mostra eficiência em operações
de transposição, rotação, inversão, ampliação e diminuição de
contornos.

\section{Contornos utilizados}
\label{sec:contorno-utilizado}

Nesta peça foram utilizados o contorno 1, representado pelas figuras
\ref{fig:contorno-1} e \ref{fig:repres-cont1}, e dois contornos (2 e
3) resultantes da alteração dos parâmetro altura e duração (veja
figuras \ref{fig:repres-cont2} e \ref{fig:repres-cont3}).

\begin{figure}
\begin{minipage}{5cm}
  \centering
\begin{verbatim}
(event 'contorno-1
 :altura 4
 :duracao 4
 :tempo 60
 (0 0)(2 3)(3 2))
\end{verbatim}
  \caption{Representação do contorno 1}
  \label{fig:repres-cont1}
\end{minipage}
\hfill
\begin{minipage}{5cm}
  \centering
\begin{verbatim}
(event 'contorno-2
 :altura 8
 :duracao 2
 :tempo 80
 (0 0)(2 3)(3 2))
\end{verbatim}
  \caption{Representação do contorno 2}
  \label{fig:repres-cont2}
\end{minipage}
\hfill
\begin{minipage}{5cm}
  \centering
\begin{verbatim}
(event 'contorno-3
 :altura 1
 :duracao 4
 :tempo 100
 (0 0)(2 3)(3 2))
\end{verbatim}
  \caption{Representação do contorno 3}
  \label{fig:repres-cont3}
\end{minipage}
\hfill
\end{figure}

\section{Tipos de preenchimento}
\label{sec:tipos-de-preench}

O preenchimento dos segmentos de um dado contorno pode ser feito de
várias maneiras. Ele depende de como são utilizados os seguintes
elementos:

\begin{itemize}
\item número de notas
\item relação entre silêncio e som
\item altura
\item ritmo e duração
\item articulação
\end{itemize}

O menor número de notas que um contorno pode ter é exatamente o número
de extremidades que ele tem: três no contorno escolhido para este
trabalho (veja exemplo na figura \ref{fig:notas-extremidades}). Em
cada segmento é possível adicionar outras notas que unam as
extremidades (veja figuras \ref{fig:varias-apojatura} e
\ref{fig:muitas-notas}). O número máximo possível de notas depende da
diferença de altura e do intervalo de tempo entre as extremidades.

\begin{figure}[h]
\begin{minipage}{8cm}
  \centering
  \includegraphics[scale=1]{3notas}
  \caption{Preenchimento com notas apenas nas extremidades}
  \label{fig:notas-extremidades}
\end{minipage}
\hfill
\begin{minipage}{8cm}
  \centering
  \includegraphics[scale=1]{variasnotas-1}
  \caption{Preenchimento de segmentos com apojatura}
  \label{fig:varias-apojatura}
\end{minipage}
\hfill
\end{figure}

\begin{figure}[h]
  \centering
  \includegraphics[scale=1]{variasnotas-2}
  \caption{Preenchimento de segmentos com muitas notas}
  \label{fig:muitas-notas}
\end{figure}

O preenchimento depende também da relação entre silêncio e som de cada
segmento. É possível ter notas completamente ligadas (veja figura
\ref{fig:notas-extremidades}) ou ter interrupções em forma de pausas
(veja figura \ref{fig:alternancia}).

\begin{figure}[h]
  \centering
  \includegraphics[scale=1]{alternancia}
  \caption{Preenchimento com alternância entre som e silêncio}
  \label{fig:alternancia}
\end{figure}

Intervalos melódicos entre as notas dos segmentos interferem no
preenchimento. Um dado segmento com as duas notas de suas extremidades
e uma nota mediana pode ser dividido do ponto de vista da altura ao
meio, como no exemplo da figura \ref{fig:divisao-meio}, ou em outra
proporção qualquer, como na figura \ref{fig:varias-apojatura}. Um
exemplo de segmento com quatro notas e proporção desigual é exibido na
figura \ref{fig:divisao-desigual}.

\begin{figure}[h]
\begin{minipage}{8cm}
  \centering
  \includegraphics[scale=1]{divisao-meio}
  \caption{Preenchimento de segmento com 3 notas com proporção de 1:1}
  \label{fig:divisao-meio}
\end{minipage}
\hfill
\begin{minipage}{8cm}
  \centering
  \includegraphics[scale=1]{divisao-desigual}
  \caption{Preenchimento de segmento com 4 notas e proporção desigual}
  \label{fig:divisao-desigual}
\end{minipage}
\hfill
\end{figure}

A influência da duração no preenchimento é semelhante à dos intervalos
melódicos. O ritmo e a articulação também podem ser utilizado para
diversificação do preenchimento (veja figuras \ref{fig:ritmo} para
ritmo e \ref{fig:varias-apojatura} para articulação).

\begin{figure}[h]
  \centering
  \includegraphics[scale=1]{ritmo}
  \caption{Diversidade de preenchimento dada pela variação rítmica}
  \label{fig:ritmo}
\end{figure}

\subsection{Combinações de elementos}
\label{sec:comb-de-elem}

Estes elementos podem ser combinados. A figura
\ref{fig:varias-apojatura}, por exemplo, contém um fragmento de dois
segmentos com preenchimentos diferentes. No primeiro há uma nota de
ligação entre as extremidades com proporção de altura e duração
desigual, som contínuo e articulação legato. No segundo segmento há
apenas as notas das extremidades, som interrompido por pausa e
articulação staccato.

\section{Forma}
\label{sec:forma}

Formalmente esta peça está dividida em três seções. A característica
de cada seção é o contorno utilizado e o seu andamento. Na primeira
seção é utilizado o contorno 1, na segunda, o contorno 2 e na terceira
o contorno 3. Os andamentos são definidos nas representações dos
contornos.

\section{Uso de outros elementos}
\label{sec:uso-de-outros}

O uso de inversões, rotações e combinações de operações de contornos
foi evitado para privilegiar o preenchimento dos contornos. Foram
utilizados elementos como variação de dinâmica, de agógica e simples
transposições tomando como base os pontos extremos dos contornos. O
contorno final é o único diferente dos representados pela figura
\ref{fig:contorno-1}.

\section{Documentação da composição}
\label{sec:docum-da-comp}

Todos os contornos utilizados foram marcados com comentários no código
fonte da obra, que foi editada com o software livre \eng{Lilypond}
\cite{Nienhuys2007}. Isso permite a precisão na localização de cada
contorno. (veja excerto do código na figura \ref{fig:codigo}).

\begin{figure}[h]
  \centering
\begin{verbatim}
segA = {
  \textSpannerUp
  \override TextSpanner #'edge-text = #'("accel " . "")
  \relative c {
    %% contorno 1: segmento 1 em 3 notas (apoj). segmento 2 em 2 notas
    f2\pp\<\( \acciaccatura e'8 f8-.\)\mp r cis4~\pp\<
    %% contorno 2: segmento 1 em 3 notas. segmento 2 em 3 notas (apoj).
    cis8\! d~\p d4~ d8 \breathe bis'\( cis4 \acciaccatura b!8 a4~
    %% contorno 3: segmento 1 em 6 notas (ritmo variado e
    %% apoj). segmento em 5 notas (ritmo variado).
    a16\) b8\<( cis16~ cis f8.) \acciaccatura g8 a32\mf( aes)
g8-. ges16-. f-. r8.
    %% contorno 4: segmento 1 em 12 notas. segmento 2 em 2 notas
    f,16->\pp\< fis g gis a32 ais b c cis d dis e f8-.\f r cis-.\p r
    %% contorno 5: segmento 1 em 2 notas. segmento 2 em 2 notas. notas curtas
    f,,8-. r r4 f'8-. r cis4\espressivo r4
    %% contorno 6: segmento 1 em 2 notas. segmento 2 em 2 notas. notas longas
    cis2\p\<( cis'4\f a)
    %% contorno 7: segmento 1 em 3 notas (apoj). segmento 2 em 2
    %% notas. notas curtas
    \setTextCresc
    f,8-. r r4 \acciaccatura e'8\p\( f4\< cis\)
    %% contorno 8: segmento 1 em 4 notas (uma no meio e
    %% apoj). segmento 2 em 2 notas. notas curtas
    f,8.-. d'16 r4 \acciaccatura e8\( f4 cis\)
    %% contorno 9: segmento 1 com repetição da primeira nota. duas no
    %% meio e apoj. segmento 2 em 2 notas.
    f,16 f\startTextSpan r8 r16 d' c8-. \acciaccatura e8 f8-. r cis-. r
    %% repetição contorno 7
    f,8-. r r4 \acciaccatura e'8\( f4 cis\)
    f,8-. r r4 \acciaccatura e'8\( f4 cis\f\)\stopTextSpan r2\fermata
  }
}
\end{verbatim}
  \caption{Código fonte da peça}
  \label{fig:codigo}
\end{figure}

\bibliographystyle{mdschicago}
\bibliography{mestrado}

\end{document}